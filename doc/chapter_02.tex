\chapter{Theoretical Literature}\label{ch:2}

Advances in technology are responsible for much of the rapid growth in incomes over the course of human history. From advances in metallurgy millenia ago to the relatively recent development of mobile telephony, technological change is disruptive: it creates winners, but it also creates losers as investments in yesterday's technology are rendered obsolete.

This chapter reviews the literature on technical change, as it relates to workers' wages in the face of recent technical change. We stay mostly within the bounds of the neoclassical school. We begin with growth models, that consider technology as a uniform force upon all parts of the labor force. We then discuss models for skill-biased change, and in particular the `canonical model.' Next, we consider the `task approach', which adapts the neoclassical model to allow for competition between human workers and capital. Finally, we discuss the Roy model and `Ricardian' models of the labor market.

\section{Early Treatments of `Technology'}

Most early treatments of technical change in the economic literature assume `technology' has a uniform impact across all types of production. For example, \citet{Ricardo1819} consider two types of innovations: landsaving innovations, that increase the output of every grade of land equiproportionally, and capital-and-labor saving innovations, that scale the output of capital and labor inputs evenly across the economy. 

More recent examples of technology that act across the whole economy are found in the growth accounting literature. Consider, for example, the neoclassical model of growth, that views production through the lens of the neoclassical production function, a kind of black-box function that `converts' inputs of capital and labor into an output good. Most formulations of the production function include a `productivity' parameter, that governs the rate at which factors of production are converted into outputs. Solow's (\citeyear{Solow1957}) well-known functional form, \begin{equation}\label{eq:solow}
F(K,L,t)=A(t)f(K,L),
\end{equation}
included measures for capital ($K$) and labor ($L$) inputs, but also allowed `technology' ($A(t)$), which he called total factor productivity (TFP), to vary over time. He deliberately left the definition of TFP vague, to simply mean any change in the rate of production: ``all sorts of things will appear as technical change'' \citep[p.312]{Solow1957}. TFP, then, was whatever was not already accounted for by measured capital and labor.

To estimate TFP, \citet{Solow1957} rearranged \eqref{eq:solow} and substituted US national accounting statistics from 1909 to 1949 for real GDP, capital and labor. The resulting estimates of TFP increased more or less monotonically over the first half of the 20th century, and by 1949 $A$ had grown to about double its initial value.

Today, the neoclassical approach to growth accounting remains an important field of study for understanding the interaction of `technology' and income.  \citet{Mankiw1992}, for example, demonstrated that, a Solow model augmented to include a measure of human capital as a factor of production, can explain cross-country variation in the level of income very well indeed. However, as elegant and convenient as the neoclassical growth model is, and notwithstanding its success at explaining the dispersion of incomes between countries, it does not explain the evolution of wage profiles within countries. In particular, it has not been able to explain the secular trend of increasing income inequality over the past 30 years in developed countries. 

\section{Rising U.S. Education Premia}\label{sec:risingpremia}

For much of the 20th century, wage growth differentials between different skill and educational groups in the United States had remained more or less stable. However, beginning in the 1980s, empirical evidence showed that the wages of skilled workers had begun to grow faster than those of unskilled workers \citep{Juhn1993}. At the same time, the supply of skilled workers in the United States, relative to unskilled workers, had grown dramatically. These empirical regularities suggested that, since firms were demanding increasing amounts of high-skilled labor, even at increasing wage rates, the productivity of skilled workers had increased, relative to that of unskilled workers. The existing neoclassical model could shed no light onto this trend, and a more nuanced understanding of the relationship between technology and productivity was required.

One set of explanations pointed to the changing nature of labor market institutions. \citet{Freeman1994}, for example, suggested that about 50 per cent of the increase in the `white collar' premium paid to U.S. men could be explained by the decline in the unionization rate. If it is accepted that that union bargaining activities result in a narrowing of wage dispersion between unionized workers, then a decrease in the union rate should result in a loss of worker bargaining power, and thus a decrease in wages. But since unions tend to cover mainly `blue-collar' occupations, then this trend should result in a widening gap between blue- and white-collar work. Indeed, recent studies of the relationship between the union membership rate and the dispersion of income lend this explanation solid empirical support, both in the United States \citep{Card2004,DiNardo1996} and Australia \citep{Borland1996}.

Other institutional explanations for the growing premium paid to skilled workers include sociological changes, such as changes in norms associated with worker pay \citep{Mitchell1989}. Another quite plausible explanation for the observed trends, is completely independent of any qualitative change in the nature of technology or jobs. According to this view, the rise in the relative demand for skilled labor, and concomitant decline in the density of union workers,  simply reflected broader economic trends, as changes in the composition and distribution of production activity.

One economic trend that could bring about a change in the demand for skilled and unskilled labor is the transfer of many US manufacturing jobs overseas. This argument was advanced by \citet{Murphy1992}, who appeal to changes in the demand structure of US labor as a consequence of trade and competition with overseas producers. The patterns of U.S. trade shifted over the 1980s from trade surplus to deficits, favoring the manufacture of goods in low-cost countries, instead of with domestic, high-cost workers. Murphy {\em et al.} study the relationship between wage rates for white males between 1963 and 1979, and macroeconomic measures of output and wages both in the U.S. and overseas, concluding that changing patterns of trade go a long way to explain changes in the wage structure.

During this debate, some authors argued that technological progress might have reduced demand for certain kinds of physical work by enabling the substitution of capital for labor. Using logitudinal data gathered from census files, \citet{Davis1991} argue that, through capital investment in automated equipment and machinery, the manufacturing sector replaced labor-intensive jobs with plant capital. Under this explanation, growing wage differentials between college-educated workers and high school graduates is a result of changes in the demand for labor, as the demand for non-manual work increases, and the demand for manual labor softens. \citet{Krueger1993} expanded on this argument: using microdata on individual workers, he found a premium associated with those occupations that involve computer use.\footnote{For a rather sardonic refutation of Krueger's position, see \citet{DiNardo1997}, who show that an effect of a similar magnitude is associated with occupations that employ the use of lead pencils.} Kreuger's observation that wage changes are due to the {\em nature} of the work (the fact that computers are used), rather than its outputs or the skills required to undertake it, was to become an important consideration in future work.

\section{Models of Skill-Biased Technological Change}

The explanation for rising skill premia that won acceptance in the literature is that new technologies, emerging over the post-war era, are complementary to skilled work, but not other types of labor. Based on US manufacturing data, \citet{Griliches1969} proposed models of labor-augmenting technology, which he called `capital-skill complementarity.' The modern form of the `canonical' SBTC model is due to \citet{Tinbergen1974,Tinbergen1975}, who developed a model of the labor market where different kinds of labor were factors in production. His model, which included university graduates and unskilled workers, employed the familiar CES production function of the form,
\begin{equation} F(L_1, L_2) = \left[\alpha (\beta_1L_1)^{\frac{\sigma-1}{\sigma}} + (1-\alpha) (\beta_2L_2)^{\frac{\sigma-1}{\sigma}}\right]^{\frac{\sigma}{\sigma-1}}, \end{equation}
but allowed each type of labor to have different levels of efficiency, $\beta_i$.

Naturally, models of the division of labor are not new: since the time of Adam Smith, it has been well known that the productivity of modern production processes, at any scale, depends crucially on the specialization of labor into a number of different jobs. In Adam Smith's famous study of a pin factory, Smith observed somewhere between a 240-- and 4800--factor increase because of complementarities between different, specialized jobs \citet[I.3]{Smith1776}. Tinbergen recognized that it would be unwieldy to attempt to model every type of job using a production function, as the dimensionality of the model would quickly explode. However, armed with the observation that many properties of jobs are highly correlated, he proposed that a good characterization of the labor market can be made by singling out just one or two properties of different jobs. One particularly relevant property of jobs is the degree of {\em schooling} an individual has received. \citet{Tinbergen1974} modeled graduate and unskilled workers as imperfect substitutes in production, to capture the fact that firms typically have to hire different types of workers---and so that changes in the productivity of one type of labor affects demand for {\em all} types of labor.

Tinbergen identified two implications of his model. First, since the higher-productivity `graduate' workers were scarce relative to the supply of unskilled workers, graduates would be able to charge employers `scarcity rents,' as well as additional rents for their individual productivity. In the medium run, the availability of these rents would induce a `race' for investment in education, as unskilled workers seek to increase their human capital in order to access these rents.

Tinbergen's model was further developed by \citet{Katz1992} and others, and brought to bear on the empirical regularity of rising educational returns in the United States and elsewhere, beginning in the 1980s. In their model, they suggest that new workplace technologies disproportionately complement highly-skilled technical and managerial jobs, relative to low-skilled manual and service jobs. Under this explanation, the premium paid to high-skilled labor increases for two reasons: first, since high-skilled workers become relatively more productive, wages to high-skilled occupations are higher at the margin. There is also evidence that, in the United States at least, an increase in the demand for skilled labor, relative to its supply, has resulted in higher wages for skilled occupations. Such technologies are said to exhibit \emph{skill bias} \citep{Autor2006}.

In addition to the rising skill premium, and the observation that it was high-skilled workers that benefitted from the new computing and telecommunications technology appearing in the 1980s, there was `virtually unanimous agreement' that SBTC was responsible for the increase in demand for high-skilled workers \citep[p.41]{Johnson1997}. A wealth of empirical evidence was accumulated, at the industry and country level \citep{Berman1994,Autor1998,Berman1998}, and at the firm and plant level \citep{Levy1996,Bresnahan2002}. However, there nonetheless remained dissenting voices, which we briefly review below (\S\ref{sec:dissent}).

\section{The Canonical Model}\label{sec:canonical}

The formulation of the SBTC model that gained wide acceptance in the literature, dubbed the `canonical model' by \citet{Acemoglu2011}, was adopted by a large number of authors analyzing skill-biased technology \citep[][e.g.]{Katz1992,Katz1999,Goldin2007,Acemoglu2011}.
We will briefly outline the main features and implications of this model, following the notation employed by \citet{Acemoglu2011}.

The canonical model imagines an economy where the only inputs to production are two types of workers, those with `high' skill and those with `low' skills, that work together to produce a single output good. These two types of workers are imperfect substitutes, so that although firms require both types of workers, they will select the mix of labor they demand based on their relative efficiency. Both types of workers employ a generic `technology', similar to Solow's TFP, that represented by a single number. This number linearly scales their output, determines how efficient their labor is. Although this isn't important for the main findings, the model is flexible enough to consider variety within the workforce. Workers are paid according to how much they individually contribute to production, and each individual has a different levels of productivity. We can then think about the total amount of productive effort contributed by both types of labor as our inputs to production.

Formally, consider a competitive economy with two different, imperfectly substitutable types of labor: high-skilled and low-skilled. Workers are heterogeneous, with different levels of efficiency within each skill group. Denote the efficiency distribution of workers in the high- and low-skilled groups $\sH$ and $\sL$, respectively, so that worker $i$ supplies $h_i\in\sH$ efficiency units, and worker $j$ in the low-skilled group supplies $\ell_j\in\sL$. Let the total supply of each type of labor be $H$, and $L$, respectively, where
$$
    H = \int_{h_i \in \sH} h_i\ di \quad\text{and}\quad L = \int_{\ell_j \in \sL} \ell_j\ dj,
$$
and both types are paid the same wage per efficiency unit, respectively $w_h$ and $w_\ell$. Production in this economy is governed by a constant elasticity of supply (CES) aggregate production function,
\begin{equation}  \label{eq:prod}
Y = \left[
  \left(A_LL \right)^\frac{\sigma-1}{\sigma}
  +
  \left(A_HH \right)^\frac{\sigma-1}{\sigma}
  \right]^\frac{1}{\sigma-1},
\end{equation}
where the elasticity of substitution is $\sigma>1$, and the coefficients $A_L$ and $A_H$ represent the `technology' that governs the efficiency of each type of worker. Below, we will alter these parameters to conduct experiments on the impact of technological change on wage levels.

For our purposes, we are interested in two claims about relative wages made by this model: that neither technological improvements, nor a generalized shift from low-skilled to high-skilled work should ever cause low-skilled wages to decrease, and that SBTC should result in a monotonic increase in wages across the skill spectrum. To see this, we will first derive the expressions for the equilibrium wage for each type of labor. Since the economy is competitive, unique equilibrium wages for both both high- and low-skilled workers are given by their respective marginal products. Wages can therefore be found by differentiating \eqref{eq:prod} with respect to labor supply:
\begin{align}
w_h &= \frac{\partial Y}{\partial H} 
     = A_H^\frac{\sigma-1}{\sigma}\left(
              A_L^{\frac{\sigma-1}{\sigma}} (H/L)^{-\frac{\sigma-1}{\sigma}} + A_H^{\frac{\sigma-1}{\sigma}}
        \right)^{\frac{1}{\sigma - 1}} \label{eq:wh} \\
w_l &= \frac{\partial Y}{\partial L} 
     = A_L^\frac{\sigma-1}{\sigma}\left(
              A_L^{\frac{\sigma-1}{\sigma}} + A_H^{\frac{\sigma-1}{\sigma}}(H/L)^{\frac{\sigma-1}{\sigma}}
        \right)^{\frac{1}{\sigma - 1}} \label{eq:wl}
\end{align}
The first claim follows from differentiating these wage equations. Notice in \eqref{eq:wl} that $\partial w_l/\partial A_H \geq 0$. This means that, in this model, an increase in technology for high-skilled workers does not reduce the wage for low-skilled workers. Technological progress should in fact result in positive wage improvements for both high- and low-skilled workers. 

Next, it can be shown that $\partial w_l/\partial(H/L)>0$. An increase in the relative supply of high-skilled workers, $H/L$, should therefore not decrease the wage of low-skilled workers. Rather, as high-skilled work becomes more productive and the ratio of skilled to unskilled workers increases, the demand for low-skilled work simultaneously increases. 

Regarding the second claim, consider the wage ratio between high- and low-skilled labor, $\omega=w_h/w_l$ (for convenience, we will consider the log ratio.) It is straightforward to show that this ratio depends on the state of technology and labor inputs:
\begin{equation}\label{eq:omega}
\log \omega = \frac{\sigma-1}{\sigma}\log\left(\frac{A_H}{A_L}\right) - \frac{1}{\sigma}\log\left(\frac{H}{L}\right).
\end{equation} % insert discussion of value of \sigma
This equation illustrates the two countervailing forces of Tinbergen's (1974) `race' for education that govern the magnitude of the skill premium. Holding the labor supply ratio constant, and recalling our assumption that $\sigma >1$, an increase in skill-biased technology $A_H/A_L$ results in a larger $\log\omega$. On the other hand, holding technology constant, an increase in the proportion of workers providing high-skilled labor should decrease the log skill premium.\footnote{Formally, $\partial \log\omega / \partial(A_H/A_L) > 0$, and 
$\partial \log\omega / \partial(H/L) < 0$.} In this model, a rising skill premium occurs when the first term of \eqref{eq:omega}  dominates the second.

% Put MOAR in here -- include secular trend of rising technology

To review, the SBTC model claims that unless there is technical regress, wages for all skill types will always increase, and never decrease (wages should follow a monotonic path.) Second, in the presence of an increasing proportion of workers conducting skilled work, the model is consistent with either a rising or a falling log skill premium.

\section{Alternative Perspectives on SBTC}\label{sec:dissent}

A crucial assumption of the SBTC model is that, like the \citet{Solow1957} growth accounting scheme, all technological change is treated as exogenous. One alternative to the canonical model of exogenous skill-biased technological change is presented by \citet{Beaudry2005}. They consider technical change as a discrete event, and model two `modes of production' as two separate production functions: the `old' and the `new.' In their model, the transition to the new technology is gradual, as capital of the old type is gradually replaced by the new. Importantly, their model implies that any inequality caused by the change in technology should eventually {\em narrow} as more capital is invested in the new technology, and the economy switches over to the new mode of production. At the point of the new technology's invention, only a small fraction of the economy is earning higher incomes by exploiting the new technology. But as the transition to the new type of technology completes, this difference should fade completely. Applied to skill-biased technology that exhibits capital-skill complementarity, this model implies that further investments in computer technology should actually {\em decrease} between-group inequality.

\citet{Card2002} criticize the broad acceptance of SBTC: they concede that technology is a source of changes to the wage distribution, but argue that its importance as a driver of inequality is overblown. They argue that the coincidental timing of the rise in inequality and the emergence of the personal computer gave undue salience to SBTC, overshadowing other explanations such as the decline in union membership and the transfer of manufacturing jobs to outside of the United States.

The most important critique of SBTC was, in fact, a refinement of it. \citet{Levy2003} pointed out that the notion of `skill' was unhelpful in a model of technical change arising from computers and telecommunications technology. They argue that, although computers are a complement to certain occupations that involve a high degree of cognitive work, they tend to be a substitute for other types of routine activities, such as filing clerks and salespersons. Computers, despite their sophistication, are only capable of performing a very limited set of simple, routine tasks. They excel at processes which require calculation and simple symbolic manipulation, and are not prone to the same types of errors as human workers. It is this fact which has led to their widespread adoption in a wide range of electronic service delivery such as ATMs that were formerly the domain of human personnel. Yet, any task that requires abstract thought or physical coordination, however elementary they may appear to a human worker, is out of reach for a computer. Activities such as stacking shelves or driving a taxi are areas in which, for the moment at least, human workers enjoy a competitive advantage. 

\section{The `Task Approach'}

The approach taken by \citet{Levy2003}, and the literature that followed, differs from the neoclassical approach to production in a fundamental way. The neoclassical production function, which views aggregate economic output as a simple function of inputs such as capital and labor, does not consider the specifics of the processes which produced that output \citep{Acemoglu2011}. Although the canonical approach has been very successful in explaining aggregate output levels, it is not sensitive to qualitative changes in the nature of production such as changes in the technology which produce output:
\[ \text{capital, labor} \quad \overset{F(\cdot)}{\xrightarrow{\hspace*{2cm}}} \quad \text{output}. \]
The {\em task approach} presents an alternative perspective. Rather than viewing output as a direct function of resource inputs, as in the neoclassical approach, it includes the tasks performed by labor and/or technology as an additional layer of indirection between factors and production. Under this setup, the same tasks can be performed either by capital or labor, and the two factors can even compete for the role of performing certain tasks. It then becomes the domain of the economic model to explain which factors were assigned to which tasks \citep{Autor2013,Acemoglu2011}:
\[ \text{capital, labor} \quad \overset{\text{assignment}}{\xrightarrow{\hspace*{2cm}}} \quad \text{tasks} \quad \overset{F(\cdot)}{\xrightarrow{\hspace*{2cm}}} \quad \text{output}. \]

By separating the factors that produce tasks, and the tasks required for production, this approach facilitates the inclusion of worker \emph{skills} in the model. For the purposes of this analysis, we follow \citet{Autor2013} in viewing a \emph{task} as a discrete unit of work, which may be used to create final goods and services, and a worker's \emph{skill} as the stock of capabilities for the execution of those tasks. Importantly, under this framework, the allocation of workers' skills to tasks is considered endogenous to the model: heterogeneous workers apply their skills to tasks where they enjoy a competitive advantage.

Under this framework, the performance of tasks is not confined to human workers. Since the industrial revolution, investments in labor-saving capital by firms has seen a dramatic change in the performance of repetitive tasks. The pace of technical change has been continual: as automated looms replaced hand-weavers in the 18th century, so too are cheap computers replacing administrative clerks and service workers in the 21st century. As \citet{Brynjolfsson2011} point out, there is no economic reason to expect that, as jobs formerly performed by humans are replaced by computing capital, that new opportunities for workers skilled in that type of labor will arise. The phenomenon of firms substituting capital equipment for repetitive human labor was the driving force behind the industrial revolution \citep{Goldin1998}. There is no reason to expect that the present trend of wholesale substitution of capital for human labor will not continue.

The level and price of task-performing labor can be viewed as an outcome of the demand for particular tasks from workers and machine capital, and the supply of task-performing labor and capital. Unlike the canonical model, where technology is viewed as factor-augmenting,  technology can therefore be viewed as substitutes for some tasks, and complements for others. Thus, firms are able to substitute between capital and human workers for the execution of certain tasks.

In recent decades, the most important source of labor-saving capital has been information and computer technology (ICT). As the real cost of computation has fallen precipitously over the 20th century, computers have been able to execute a wider range of tasks at a lower cost. In the presence of falling costs of ICT, the question of work force polarization can thus be framed as an outcome of a decline in the real cost of computing capital, relative to the wage cost of human workers performing similar tasks.

It is therefore plausible, that the widespread adoption of ICT is a major driver of compositional changes in the workforce. 

Since the late 1990s, both in Europe and the United States, the data show a marked polarization in the work force \citep{Goos2007, Autor2006}. This polarization has simultaneously manifested in \emph{wages} and in \emph{jobs}: both wage growth and growth in the level of employment are concentrated in high-skill jobs, and to a lesser extent, the bottom end of the skill spectrum. Middle-skill jobs have stagnated since the 1990s, both in terms of remuneration and level. The recent rise of ICT investment by firms has been attributed to this trend, both because many middle-skilled jobs can be substituted by computer capital, and because communications technologies enable firms to outsource non-customer-facing roles to remote locations in order to take advantage of cheaper labor.

\section{Capital-Labor Substitutability}

To operationalize their concept of separate tasks and factor inputs, \citet{Levy2003} propose a simple model where the inputs to production are two types of tasks: `routine' and `nonroutine.' In this context, `routine' has a very particular meaning: it refers to tasks that can be easily codified into computer programs or performed by machinery, such as adding up a column of numbers, or conveying a message from one place to another. This notion of routineness differs from its usual, colloquial definition. Mundane tasks such as sweeping a floor or stacking shelves, that are not (yet!) candidates for replacement by machines, are not `routine' under this definition. Nonroutine tasks include all other tasks, including cognitive tasks, such as high-skill professional and managerial work, and low-skill manual work, where physical coordination and strength are an important part of the job. 

The linkage between routine and nonroutine tasks is simple. There are three factors of production: computing capital ($C$), routine labor ($L_R$) and nonroutine labor ($L_N$), where all three are measured in efficiency units. Nonroutine tasks can only be performed by labor (their relationship is one-to-one), and routine tasks can be performed {\em either} by routine labor, or by computer capital. \citet{Levy2003} employ a Cobb-Douglas production function:
\begin{equation}\label{eq:alm}
  F(L_R, L_N, C) = \left(L_R + C\right)^{1-\beta}\left(L_H\right)^\beta.
\end{equation}
Since routine labor and computing capital are substitutes, \eqref{eq:alm} implies that under competitive conditions, the price of routine labor is pinned down by the cost of computer capital. \citet{Levy2003} show that, at equilibrium, a decrease in the cost of computer capital (and hence routine labor), will cause the demand for routine task inputs to increase, as firms substitute towards the cheaper factor of production. As a result, the level of production increases. Since routine and nonroutine tasks are imperfect substitutes in production, a decrease in the cost of routine task inputs also causes an increase in the demand for non-routine tasks. So, if the supply of non-routine labor is fixed, then the relative price paid to non-routine labor increases.

To summarize, as the cost of computing capital decreases, the wage rate and demand for routine labor will decrease, and the demand for and wage paid to non-routine labor will increase. \citet{Levy2003} posit that, if individuals have heterogeneous allocations of skills, then those individuals in this economy would increasingly choose to supply non-routine labor, according to their comparative advantage. In this sense, the model can be described as `Ricardian,' and bears similarity to Ricardian trade models, since individuals compete according to their comparative advantage.

The relatively informal model of technology-skill substitutability developed in \citet{Levy2003} offers a useful explanation of the evolution of wages for routine jobs. However, it is simple, and does not consider general equilibrium effects, nor is it capable of a more nuanced analysis of the real-world labor market. More comprehensive models of self-selection are associated with the assignment literature, and extend from Roy's model of self-selection, which we describe now.

\section{Roy's Model of Occupational Choice}

The economic insight behind many models of assignment step from Roy's~(\citeyear{Roy1951}) model of self-selection, where individuals are endowed with heterogeneous skills, and can select between multiple occupations according to their own comparative advantage. In its most general form, the model is sophisticated enough to handle any number of occupations, and distributions of skills. For our purposes, let us here review Roy's original simple thought experiment, which takes place in a fictitious remote village situated on the banks of a river, and near a large forest. In that village, individuals with heterogeneous skills must choose between one of two occupations: hunting rabbits and fly fishing. 

The level of skill required to practice these jobs is quite different: hunting rabbits, which are described as `slow and stupid,' is easy. As a result, the return to rabbit hunting skills is not particularly great: skilled trappers will not catch many more than unskilled trappers. Fly fishing, by contrast, is extremely difficult. In this occupation, the return to skill is considerable: unskilled fishermen will hardly catch anything, but those who have mastered the art can make a good living.

In the model, the wage accrued to each activity arises from the sale of what is caught. Both fish and rabbits fetch a well-known market price, and an individual's wage is determined simply by the product of the market price and the size of the catch. It is assumed that individuals make their labor supply decisions based only on their wage; if the distribution of each type of skill is continuous, then individual agents will (almost) never be indifferent between the two activities.

Roy's intention was to explain the {\em selection effect}, or the difference in productivity of individuals in a given occupation relative to the population mean, as a result of their own self-selection decisions. For illustrative purposes, we present here a simple parametric example with two occupations from \citet{Heckman2008}. Although this simple example has considered only two occupations, Roy's model can be generalized to any number of occupations; the intention here is to illustrate the intuition behind the model, rather than derive a general result. Assume first that individual $i$'s efficiency follows a bivariate normal distribution with covariance $\ve{\Sigma}$, where an individual would catch either $F_i$ fish, or $R_i$ rabbits, depending on the occupation selected:
\begin{equation*}
 \begin{bmatrix}log{F_i} & log{R_i}\end{bmatrix}' \sim N(\bmu, \ve{\Sigma}),
 \label{eq:dist}
\end{equation*}
where $\ve{\Sigma}$ is a $2\times 2$ matrix that is not necessarily diagonal. If the market prices for fish and rabbits are $\pi_f$ and $\pi_r$ respectively, then it can then be shown that the average productivity in each sector is
\begin{align}
 E\left[ \ln F_i | \pi_fF_i \geq \pi_rR_i \right]
   &= \mu_f + \frac{\sigma_{ff} - \sigma_{fr}}{\sigma}
     \lambda\left(
       \frac{\ln \pi_f - \ln \pi_r + \mu_f - \mu_r}{\sigma}
       \right)\label{eq:srf}
\intertext{for fishing, and}
 E\left[ \ln R_i | \pi_rR_i \geq \pi_fF_i \right]
   &= \mu_r + \frac{\sigma_{rr} - \sigma_{rf}}{\sigma}
     \lambda\left(
       \frac{\ln \pi_r - \ln \pi_f + \mu_r - \mu_f}{\sigma}
       \right)
\label{eq:srr}
\end{align}
for rabbit hunting, where $\sigma^2$ is the variance of individuals' skill ratios, $\ln(F_i/R_i)$, and $\lambda(\cdot)$ is the inverse Mills ratio.

The second terms on the right-hand sides of \eqref{eq:srf} and \eqref{eq:srr} are {\em selection effects}, and must be positive for at least one of the occupations. Specifically, the selection effect is positive for occupations with high skill variance, that is, those occupations that reward high skill levels and punish low skill levels. Whether there is positive selection into occupations with {\em lower} variance depends on the covariance with other skills ($\sigma_{fr}$ in this example.)

Equations \eqref{eq:srr} and \eqref{eq:srf} yield rather intuitive comparative static predictions in the event of a market price change for one of the goods. In the event of a price shock (which may result from a shift in either demand or supply), agents will self-select into the market where prices have increased. For example, if the relative log price of rabbits ${\left(\log(\pi_r)-\log(\pi_f)\right)}$ increases, {\em ceteris paribus}, then ${\pi_rR_i \geq \pi_fF_i}$ will be true for some proportion of marginal agents who had formerly been better off fishing. These marginal agents will transfer into the rabbit-hunting occupation, which has a secondary effect of reducing the observed wage dispersion in the fishing occupation.

This intuitive comparative static prediction forms the basis for the empirical analysis we undertake in this chapter. If the polarization hypothesis suggested by \citet{Levy2003} is correct, then the demand for routine and offshorable occupations should have decreased in the period 1981-2010. As wages fall, individuals transfer into other occupations, and consequently a decrease in both the level and dispersion of wages in these occupation should be observed.

\section{`Ricardian' Models of the Labor Force}\label{sec:ricardo}

By extending results from the assignment literature, a number of authors have developed more comprehensive models of worker self-selection in the presence of more than two or three types of labor or goods. \citet{Costinot2010} make use of a Dixit-Stiglitz production function to generalize the simple \citet{Levy2003} model to a continuum of types of workers that produce a continuum of goods in the context of international trade. In this model, and much like the simple Roy model outlined above, workers self-sort along the continuum of workers according to their own comparative advantage.

The model of Costinot {\em et al.} is modified by \citet{Acemoglu2011} to explicitly separate the roles of tasks and labor. Although we do not implicitly attempt to estimate this model in the following chapters, it is worth discussing it in some detail because its implications give a good description of the `routinization' and `polarization' hypotheses, which we do test.

\citet{Acemoglu2011} analyze an economy with a single output good, $Y$, that is produced on a continuum of tasks on the unit interval. They combine the output level $y_i$ of each task $i\in[0,1]$, where the output good is the numeraire, using a Cobb-Douglas production function:
$$
  \log Y = \int_0^1 \log y_i\ di
$$

In this model, there are three types of labor: low ($L$), medium ($M$) and high ($H$). Each type of labor, along with capital $k_i$, can perform each task $i$, according to the production function,
$$
y_i = A_L\alpha_{L,i}\ell_i + A_M\alpha_{M,i}m_i + A_H\alpha_{H,i}h_i + A_K\alpha_{K,i}k_i.
$$
Productivity schedules for each task $i$ are given by $\alpha_{L,i}$, $\alpha_{M,i}$ and $\alpha_{H,i}$. Differences in these schedules afford each worker a different comparative advantage in different tasks.

To model a spectrum of task complexity, the model assumes that complexity is increasing in the task index, with $i=0$ being the least complex task and $i=1$ the most complex. It is further assumed that $\alpha_{L,i}/\alpha_{M,i}$ and  $\alpha_{M,i}/\alpha_{H,i}$ are
continuously differentiable and monotonically decreasing, and that ${\alpha_{L,i} \leq \alpha_{M,i} \leq \alpha_{H,i}.}$
Even though high-skilled workers enjoy an absolute advantage over medium-skilled workers, and similarly medium-skilled workers over low-skilled workers, {\em comparative advantage} remains, and determines the allocation of tasks among workers.

\citet{Acemoglu2011} show that, as an outcome of self-selection, an equilibrium exists and is stable. They further show that, in equilibrium, boundaries $I_H$ and $I_L$ will emerge on the unit interval, such that high-skilled workers will perform tasks where $i \in (I_H,1]$, medium-skilled workers will perform tasks where $i \in [I_L, I_H]$ and low-skilled workers will perform tasks where $i \in [0, I_L)$. Relative wages then depend on labor supply and the location of the task thresholds, which in turn depend on the comparative advantage parameters:
\begin{align*}
  \frac{w_H}{w_M} &= \left( \frac{1-I_H}{I_H - I_L} \right)\left(\frac{M}{H}\right) & \text{and} &&
  \frac{w_M}{w_L} &= \left( \frac{I_H-I_L}{I_L} \right)\left(\frac{L}{M}\right).
\end{align*}

The comparative statics of the model accord with what one might intuitively expect. In the event of a rise in the high-skilled technology $A_H$, {\em ceteris paribus}, the fraction of tasks performed by high-skilled labor increases ($I_H$ decreases), and the relative wage rates $w_H/w_M$ and $w_H/w_L$ increase. However, $w_M/W_L$ decreases, because $H$ and $M$ are closer substitutes than $H$ and $L$. Correspondingly, an increase in the high-skilled labor supply $H$, {\em ceteris paribus}, increases the fraction of tasks performed by high-skilled labor, but in this case the relative wage ratios $w_H/w_M$ and $w_H/w_L$ {\em decrease}.

The model can be extended to consider labor-replacing capital, by introducing capital that competes with one or more of the types of task inputs in the model. In the case of the \citet{Levy2003} hypothesis, this capital would compete with the middle-skilled labor, $M$. The model predicts that, in this case, the range of tasks performed by middle-skilled labor decreases, so that the middle-skilled labor supply, $M$, decreases overall. However, the presence of a competing technology places pressure on the margins of middle skilled work, $I_L$ and $I_H$. The relative movement of these margins depend on the relative productivity of high- and low-skilled labor at performing marginal tasks, relative to the displaced medium-skilled workers. If middle-skilled labor holds a comparative advantage over low-skilled workers, then low-skilled workers will be displaced, and the high-low wage ratio $w_H/w_L$ will increase.

The `polarization' hypothesis, in this model, corresponds to the presence of a labor-replacing technology in the middle of the skill distribution, and a comparative advantage for high-skilled workers relative to low-skilled workers at the margins with the middle-skilled technology. We expect to observe an increase in low- and high-skilled relative wages, but a larger increase in the high-skilled wage rate. We further expect to observe a sinking share of workers supplying middle-skilled labor, and those displaced workers moving {\em down} the task distribution.

A similar prediction applies for the displacement of workers by offshoring, since it is workers in the middle of the skill distribution whose jobs are replaced. In this case, though, workers are not replaced by technology, but instead by workers in foreign countries, who perform their jobs remotely via telephone or computer networks.

\section{Summary}

In this chapter, we have considered models of the relationship between technology and income of increasing detail. We saw that, in the growth literature, little allowance is made for different types of work, and as such `technology' is assumed to operate evenly over the entire work force. These models are unable to assess the impact of skill-biased technology.

Models of changes in the wage profile, inspired by the rising differential between college and non-college educated workers in the United States. The most widely accepted of these, the `canonical' model of skill-biased technical change, considers an economy with two types of labor that are imperfect complements. The model predicts that, in the face of increasing technology for high-skill workers, that wages will rise for both worker types, and that wages should be monotonic increasing over time, as well as across the skill spectrum.

Finally, we considered more nuanced models of technical change, where the discriminating factor was not `skill', but rather the nature of the job. In doing so, we discussed the `task approach', where occupations are considered in terms of their activities, and some activities are ripe for replacement with computers. Under the routinization hypothesis, jobs in the middle part of the skill spectrum (especially clerical and sales work) are candidates for routinzation and replacement. Similarly, the same sorts of jobs should be candidates for replacement with foreign labor, by outsourcing.

In the next chapter, we will review some of the empirical evidence for changes in workers' wages, both overseas and in the Australian labor market.

%%% Local Variables: 
%%% mode: latex
%%% TeX-master: "paper"
%%% End: 
