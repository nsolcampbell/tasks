\chapter{Conclusions \& Further Research}\label{ch:5}


\section{Main Contributions}

The main empirical finding in this study is that there is evidence for structural changes in the wage distribution in the 1980s and 1990s, and during the 2000s for the top 10 per cent of workers. In particular, we find some evidence that this structural change is due to technology shifts, and in particular we find limited support for the routinization hypothesis in the 1980s and 1990s. However, consistent with other literature on SBTC, the Australian experience appears to be somewhat different to that of other advanced countries. We also established that the relationship between skill profiles and technical change is not simple: the impact of technical change is quite different in different parts of the income distribution.

This study confirmed and updated the finding that the `canonical' model does not fit the Australian data particularly well. It does not explain the growing wage inequality observed in the data, even though rising educational attainment suggests that a skill premium should be present. It is also established that the middle-skill wage share appears to be falling, an effect that is expected if the polarization hypothesis is correct, and consistent with the data in overseas economies.

Using the methods outlined in this study, we were not able to find consistent support for the outsourcing hypothesis. This is likely due to inappropriate application of task indexes, rather than evidence that outsourcing has not occurred. However, we did find evidence of rising returns to management activities that emerged mainly over the 2000s.

Finally, one contribution of this study is to establish a detailed set of task measures that can be used with Australian occupational codings, based on the US O*NET database.

\section{Limitations}

The main limitation of this research is its use of income survey data that are not consistently coded over the entire study period. Australian occupational codings have changed relatively frequently, which makes consistent comparisons of wage profiles over time especially difficult. The construction of synthetic composite occupational classifications is an imperfect way to work around this problem, since the synthetic groups are at best only approximately comparable.

One further challenge presented by the available data is that, for privacy reasons, detailed income surveys from the 1980s and 1990s do not provide detailed occupational codings. Unfortunately, this is exactly the period when structural change appears to have occurred; without better data, it will not be possible to identify with greater precision the periods in which technical change altered the wage distribution.

Our approach observed only a limited subset of changes to the wage distribution. By assuming that the work activities involved in each occupation remained static over the entire study period, we were only able to study changes in tasks between occupations, rather than within them. More detailed task measures---and in particular, a task time series---would be helpful in addressing this limitation in future work.

Furthermore, for simplicity, this study only addressed full-time employees and workers of own account, in order to ensure a consistent comparison of tasks between periods. The rise of part-time work is an important trend in the Australian labor market, and likely to be linked to skill and activity changes \citep[see][]{Esposto2012}. The inclusion of work hours in the study, which we were able to ignore by studying only full-time workers, would broaden the study to include under-employment for low-skill and casual workers \citep{Briggs2006}.

\section{Suggestions for Further Research}

The findings in this study suggest that a more detailed examination of task-level phenomena in the Australian labor market would be fruitful. A wide range of wage surveys are available, and we have employed only one in this study. A number of alternative data sources stand out. These include the Employee Earnings, Benefits and Trade Union Membership (EEBTUM) survey, available in various forms since the mid-1970s. EEBTUM would be useful because it includes not just occupational information, but also data on union membership, a variable linked to profound structural wage changes over the past 30 years. Another alternative is the Household Income and Labour Dynamics in Australia (HILDA) survey. HILDA would also be useful, even though it covers only the 2000s, because occupational data in its early years have been re-coded to be consistent since the introduction of the new ANZSCO occupational coding scheme in 2006.

As mentioned above, the use of only part-time workers excludes a large and important part of the work force. Further research that considers the relationship between technical change and part-time work would be helpful, especially since part-time workers are relatively likely to be in routine or offshoreable occupations.

Finally, the task measures used in this study employed only a small fraction of the data available in the O*NET database. The results obtained in task decompositions suggest that managerial work is an important component of structural wage changes; perhaps task measures could be constructed that examine this trend in more detail.

\section{Final Remarks}

This study was motivated by a desire for a better understanding of factors that determine workers' wages, and in particular, the way in which technology has altered those factors. Advances in technology have undoubtedly created countless opportunities for workers over many centuries, and there is little doubt about its benefits. Yet it is also disruptive, and it imposes heavy costs on incumbents whose skills have been rendered obsolete by the latest technological advance. The `creative destruction' of technological change can be an unpredictable, devastating force on workers livelihoods;  a better understanding of where it might strike is valuable information.

In this study, we have learned that many of today's routine jobs are likely to face competition from capital equipment tomorrow, with unfortunate consequences for workers. We also found that roles without a decision-making component face increasing penalties further up the wage distribution. These are important facts to inform education policy, and valuable guidance for those facing career choices now that will affect their lifetime earnings. 

Moreover, if humans and capitals are, in some sense, complements in production, then a better allocation of people and machines to roles will surely have benefits for everyone. Rather than perceive technology as some kind of invading force on routine jobs, perhaps recent technological change should instead be viewed as a shift in the optimal mix of capital and labor, that will lead to a more productive, wealthier society?

%%% Local Variables: 
%%% mode: latex
%%% TeX-master: "paper"
%%% End: 
