\chapter{Proofs}

Since the economy is competitive, the wage is given by each worker's marginal product, computed by taking the partial derivative of $Y$.
\begin{align*}
w_L &= \partial_LY \\
    &= \alpha  L^{\rho -1} \left(\gamma  \left(C^{\eta }+H^{\eta }\right)^{\rho /\eta }+\beta  (C+M)^{\rho }+\alpha  L^{\rho }\right)^{\frac{1}{\rho }-1} \\
w_M &= \partial_MY \\
    &= \beta  (C+M)^{\rho -1} \left(\gamma  \left(C^{\eta }+H^{\eta }\right)^{\rho /\eta }+\beta  (C+M)^{\rho }+\alpha  L^{\rho }\right)^{\frac{1}{\rho }-1} \\
w_H &= \partial_HY \\
    &= \gamma  H^{\eta -1} \left(C^{\eta }+H^{\eta }\right)^{\frac{\rho }{\eta }-1} \left(\gamma  \left(C^{\eta }+H^{\eta }\right)^{\rho /\eta }+\beta  (C+M)^{\rho }+\alpha  L^{\rho }\right)^{\frac{1}{\rho }-1}
\end{align*}

One way to achieve determinate comparative static predictions is to instead consider the {\em wage share}, computed as the wage bill for the labour type, divided by the total wage bill. These wage shares are:
\begin{align*}
\theta_H &= \frac{H w_H}{H w_H+L w_L+M w_M} \\
&= \frac{\gamma  (C+M) H^{\eta } \left(C^{\eta }+H^{\eta }\right)^{\frac{\rho }{\eta }-1}}{\alpha  (C+M) L^{\rho }+\beta  M (C+M)^{\rho }} \\
\theta_M &= \frac{M w_M}{H w_H+L w_L+M w_M} \\
&= \frac{\beta  M \left(C^{\eta }+H^{\eta }\right) (C+M)^{\rho -1}}{H^{\eta } \left(\gamma  \left(C^{\eta }+H^{\eta }\right)^{\rho /\eta }+\alpha  L^{\rho }\right)+\alpha  C^{\eta } L^{\rho }} \\
\theta_L &= \frac{L w_L}{H w_H+L w_L+M w_M} \\
&= \frac{\alpha  L^{\rho }}{\gamma  H^{\eta } \left(C^{\eta }+H^{\eta }\right)^{\frac{\rho }{\eta }-1}+\beta  M (C+M)^{\rho -1}}
\end{align*}
And the comparative statics are---
\begin{align*}
\frac{\partial \theta_H}{\partial C}
&= \frac{\gamma  H^{\eta } \left(C^{\eta }+H^{\eta }\right)^{\frac{\rho }{\eta }-2} \left(\beta  M (C+M)^{\rho } \left(C^{\eta } (C (-\eta )+C+M (\rho -\eta ))-C (\rho -1) H^{\eta }\right)-\alpha  C^{\eta } (C+M)^2 (\eta -\rho ) L^{\rho }\right)}{C \left(\alpha  (C+M) L^{\rho }+\beta  M (C+M)^{\rho }\right)^2} \\
&>0 \\
%
\frac{\partial \theta_M}{\partial C}
&= \frac{\beta  M (C+M)^{\rho -2} \left(\alpha  C (\rho -1) L^{\rho } \left(C^{\eta }+H^{\eta }\right)^2+\gamma  H^{\eta } \left(C^{\eta }+H^{\eta }\right)^{\rho /\eta } \left(C^{\eta } (C (\eta -1)+M (\eta -\rho ))+C (\rho -1) H^{\eta }\right)\right)}{C \left(H^{\eta } \left(\gamma  \left(C^{\eta }+H^{\eta }\right)^{\rho /\eta }+\alpha  L^{\rho }\right)+\alpha  C^{\eta } L^{\rho }\right)^2} \\
& < 0 \\
%
\frac{\partial \theta_L}{\partial C}
&= -\frac{\alpha  L^{\rho } \left(\beta  M (\rho -1) (C+M)^{\rho -2}-\gamma  C^{\eta -1} (\eta -\rho ) H^{\eta } \left(C^{\eta }+H^{\eta }\right)^{\frac{\rho }{\eta }-2}\right)}{\left(\gamma  H^{\eta } \left(C^{\eta }+H^{\eta }\right)^{\frac{\rho }{\eta }-1}+\beta  M (C+M)^{\rho -1}\right)^2} \\
& \gtrless 0
\end{align*}

%%% Local Variables: 
%%% mode: latex
%%% TeX-master: "paper"
%%% End: 
