\begin{abstract}
Could technology be responsible for part of the rise in income inequality over the past 30 years? This research is motivated by the fact that, while technology can make workers more productive, it also has the capacity to put others out of work entirely. Research from the United States and Europe suggests that technological change has indeed caused a `polarization' of the income distribution.

In this thesis, we seek to assess the evidence for polarization in Australia. We first consider the standard model of skill-biased technical change, and show that it only poorly fits the observed data. We then test for trends in different types of occupations, an approach that has been used with success in foreign labor markets. 

First, we link the wage share of middle-skilled occupations to investment in electronic and electrical capital goods. Next, we demonstrate a relationship between qualitative properties of certain jobs, and changes in the wage distribution. We find that jobs of the kind most likely to be impacted by technology, so-called `routine' jobs, have suffered the greatest decline in income over the past 30 years.
\end{abstract}
