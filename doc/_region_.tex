\message{ !name(paper.tex)}\documentclass[a4paper,11pt]{report}

\usepackage{amsfonts}
\usepackage{amsmath}
\usepackage{amsthm}
\usepackage{amssymb}
\usepackage{graphicx}
\usepackage{fullpage}
\usepackage{rotating}
\usepackage{setspace}
\onehalfspacing

\usepackage[american]{babel}
\usepackage{csquotes}
\usepackage[style=apa,natbib=true,backend=biber]{biblatex}
\DeclareLanguageMapping{american}{american-apa}
\addbibresource{Polarization.bib}

\usepackage[colorlinks,allcolors=blue]{hyperref}
\hypersetup{%
    pdftitle={Tasks, occupations, and the polarization hypothesis},%
    pdfauthor={Alex Cooper},%
    pdfsubject={An analysis of one determinant of inequality in Australia.},%
    pdfkeywords={labor, polarization, routinization, SBTC, inequality},%
}
\ExecuteBibliographyOptions{doi=false}
\newbibmacro{string+doi}[1]{%
  \iffieldundef{doi}{#1}{\href{http://dx.doi.org/\thefield{doi}}{#1}}}
\DeclareFieldFormat{title}{\usebibmacro{string+doi}{\mkbibemph{#1}}}
\DeclareFieldFormat[article]{title}{\usebibmacro{string+doi}{\mkbibquote{#1}}}

\numberwithin{equation}{chapter}

\begin{document}

\message{ !name(chapter_04.tex) !offset(-36) }
\chapter{Occupational tasks and the wage structure}

In the previous two chapters, we have seen that `canonical' model of skill-biased technical change does a poor job of explaining the evolution of wage inequality in Australia. In particular, we have seen that while growing inequality the Australian labour market has mirrored that of overseas economies, there is no empirical evidence that this has been driven by a premium paid to more educated labour. However, the evidence presented in the preceding chapter suggests that, while educational attainment may be a poor instrument for between-group inequality, there {\em may} be an association between occupational group and the widening wage distribution. This evidence suggests that properties of those occupations---specifically, whether those occupations could be out-sourced by firms or automated with capital equipment---may explain changes in the demand, and hence occupational wage, for those occupations.

In this chapter, we will attempt to formalize this analysis, using data on occupational task content compiled by the U.S. Department of Labor to determine which occupations are likely candidates for automation and off-shoring. Following \citet{Autor2012} and \citet{Fortin2011}, we assume that workers self-select into occupations based on comparative advantage, in a model reminiscent of Roy's (\citeyear{Roy1951}) model of occupational choice. Using occupational data, we can decompose the effects of occupational properties on the wage distribution. Empirically, we take as our point of departure the analysis of the occupational wage structure in the United States performed by \citet{Fortin2011}, who build on the work of \citet{Oaxaca1973} and others to decompose the impact of demographic variables and occupational tasks on the wage structure.

\section{Related Literature}

In this analysis, we follow Roy's (1951) seminal model of self-selection, which analyzes comparative advantage in occupations where individuals have heterogeneous skills, and can select between multiple occupations. We begin with an outline the model as originally laid out by Roy, and follow the notation given in \citet{Heckman2008}. As it was originally formulated, the model considers a number of heterogeneous agents who must choose between two occupations: hunting rabbits and fly fishing. 

Importantly, the skill required to practise these jobs is quite different: rabbits are `slow and stupid,' and so are relatively easy to catch. As a result, there are no particular returns to having great skill at catching rabbits, since skilled trappers will not yield many more rabbits than unskilled trappers. However, the same cannot be said for fly fishing, which is extremely difficult. In this occupation, returns to skills are large: unskilled fishermen will hardly catch anything, but those who have mastered the art can catch a great many fish.

The wage accrued to each activity arises from selling the catch. Fish and rabbits fetch prices $\pi_f$ and $\pi_r$, respectively, and the numbers of each caught by individual $i$ is $F_i$ and $R_i$. Each individual's wage is either given by $w_{fi} = \pi_fF_i$ or $w_{ri}=\pi_rR_i,$ depending on choice of occupation. It is assumed that individuals make their labour supply decisions based only on their wage. $F_i$ and $R_i$ can be considered continuous random variables, so the probability of an agent being indifferent between each occupation is zero.

An important outcome of the model is to explain the {\em selection effect}, or the difference in productivity of individuals in an occupation relative to the population mean, as a result of self-selection. To analyse this effect, suppose that efficiency in each occupation for individual $i$ is normally distributed:
\begin{equation}
 \begin{bmatrix}log{F_i} \\ log{R_i}\end{bmatrix} \sim N(\bmu, \ve{\Sigma}) 
 \label{eq:dist}\tag{1}
\end{equation}
where $\ve{\Sigma}$ is not necessarily diagonal. Roy derived an expression for the average productivity in each sector:
\begin{equation}
 E\left[ log(F_i) | \pi_fF_i \geq \pi_rR_i \right]
   = \mu_f + \frac{\sigma_{ff} - \sigma_{fr}}{\sigma}
     \lambda\left(
       \frac{log(\pi_f) - log(\pi_r) + \mu_f - \mu_r}{\sigma}
       \right)
\label{eq:sr}\tag{2}
\end{equation}
with $\sigma^2$ the variance of individuals' skill ratio, $log(F_i/R_i)$, and $\lambda(\cdot)$ is the inverse Mills ratio, a positive function. 

The expression on the right-hand of \eqref{eq:sr} is the {\em selection effect}, and must be positive for at least one occupation. Specifically, the selection effect is positive for occupations with high skill variance. Further, whether there is positive selection into occupations with {\em lower} variance depends on the covariance between skills ($\sigma_{fr}$ in this example.)

The key challenge arising out of empirical implementations of this model is identification. If the profit maximization and log-normality assumptions can be maintained, and if wages are observable in each sector, then the model is identified. However, the Roy model is frequently used to analyze the labour supply decision, where wages for the household sector are {\em not} observable. In this case, variations {\em across } markets, or variations {\em within } markets (ie across individuals) are used to identify the model.

Three applications of the Roy model are commonly cited. The first is labour supply, where a household sector (with unobserved wages) is added, and labour supply considered a decision to participate in the non-household sector. The second is education: the labour supply decision is the choice between the `high school' and `university' sectors. Profit maximization decisions can assumed based on the cost of further education, as well as the income streams arising out of higher levels of human capital. Finally, many papers use Roy models to model occupational choice, where the `sectors' are occupations derived from census or other survey data, where profit maximization depends on cost of entry (education and certification), utility (or disamentity) of the work, and the expected labour revenue.


\section{Empirical Approach}

Oaxaca \& Blinder decomposition

Conditional wage regression

\section{Data}

Having established that polarization is likely occurring, the goal of the next part of this research project is to formalize a more rigorous test for the relationship between ICT investment and polarization. One promising approach in the literature, proposed by \citet{Firpo2011}, posits that the work force behaves as a standard Roy model, where individuals choose their occupation based on comparative advantage. To decompose changes in the occupational wage structure, they exploit a technique based on influence function regression, and compare this to a quantitative measure of the task content of occupations.

\subsection{Occupational tasks: O*NET}

One step which was skipped over in the informal analysis above was the assignment of occupations into task groups, on the basis of the occupational classification scheme. If task content is to be analyzed rigorously, and in greater detail than a simple three-occupation breakdown, a quantitative view of occupational task content is required. The standard classification scheme for occupations used in Australia, ANZSCO, simply lists by name the tasks a particular job title might be required to perform. However, the U.S. equivalent, the O*NET database, includes hundreds of quantitative scales for the level of work activities, knowledge types and abilities for individuals in each of approximately five hundred occupations. The data were constructed using expert surveys, and provide a very rich source of information about the activities that workers in each occupation actually undertake. For example, for the work activity {\em analyze data}, the occupations {\em economist} and {\em surgeon} score highly (6.58/7 and 5.49/7, respectively.) But for the work activity {\em Handle moving objects}, surgeons score 3.62/7, and economists score only 0.54/7.

We have mapped the ANZSCO (and its predecessors, various editions of ASCO and the CCLO) to the O*NET data, and sucessfully constructued a skill measure series for Australian occupational classification schemes. We then apply a transformation step, described by \citet{Firpo2011}, to build composite indexes for `automation,' `offshorability', and so on. These composite indexes provide a dependent variable which, along with levels of capital investment on an industry-by-industry basis, provide a basis by which changes in the occupational wage structure can be analyzed.

Conducting this research for the Australian work force has presented many challenges, particularly when attempting to obtain appropriate data. Unlike the United States, where detailed occupational data appears to be readily available to researchers, we have not been able to obtain survey data for occupations at the four-digit level, which has meant that, when mapping between Australian classification schemes and O*NET, we have had to dramatically reduce the fidelity of our dataset. In general, occupation variables have only been available at the one- and two-digit levels. Unfortunately, comparisons at the two-digit level cannot be made, because during our period of interest of 1981-2010 the ABS has used four different occupational classification schemes. Regrettably, there is no satisfactory way to map between these schemes in a way that is completely comparable, so comparisons must be performed at a higher level of aggregation. For the second part of this study, we are investigating the use of census data instead, for which it may be easier to obtain four-digit data. 

The decision to use census data was particularly difficult, because this new data brings with it new challenges. The key advantage of the SIH is that the survey is administered by expert interviewers, who are trained to ensure that the income reported by each respondent fits the survey criteria. The resulting income series is of high quality, and is also provided as a continuous variable, so that detail quantile measurements can be made. In the census, respondents do not provide their actual income; instead, income levels are self-reported in binned intervals. Not only does this reduce the accuracy of any analysis performed using census data, but it also necessitates more complicated estimators for changes in the occupational wage structure.


\section{Results and discussion}

%%% Local Variables: 
%%% mode: latex
%%% TeX-master: "paper"
%%% End: 

\message{ !name(paper.tex) !offset(-56) }

\end{document}

%%% Local Variables: 
%%% mode: latex
%%% TeX-master: t
%%% End: 
