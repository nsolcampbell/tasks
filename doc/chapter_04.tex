\chapter{Tasks and wages}

In the previous two chapters, we have seen that the `canonical' model of skill-biased technical change does a poor job of explaining the evolution of wage inequality in Australia. In particular, while growing inequality the Australian labour market has mirrored that of overseas economies, there is no empirical evidence that this has been driven by a premium paid to `educated' workers, relative to less educated workers. However, the evidence presented in the preceding chapter suggests that, while educational attainment may be a poor instrument for between-group inequality, there {\em may} be an association between occupational group and the widening wage distribution. This evidence suggests that properties of those occupations---specifically, whether those occupations could be out-sourced by firms or automated with capital equipment---may explain changes in the demand for, and hence occupational wage of, for those occupations.

In this chapter, we will attempt to formalize this analysis, using data on occupational task content compiled by the U.S. Department of Labor to determine which occupations are likely candidates for automation and off-shoring. Following \citet{Autor2012} and \citet{Fortin2011}, we assume that workers self-select into occupations based on comparative advantage, in a model reminiscent of Roy's (\citeyear{Roy1951}) model of occupational choice. Using occupational data, we can decompose the effects of occupational properties on the wage distribution. Empirically, we take as our point of departure the analysis of the occupational wage structure in the United States performed by \citet{Fortin2011}, who build on the work of \citet{Oaxaca1973} and others to decompose the impact of demographic variables and occupational tasks on the wage structure.

\section{Related Literature}

In this analysis, we follow Roy's (1951) seminal model of self-selection, which analyzes comparative advantage in occupations where individuals have heterogeneous skills, and can select between multiple occupations. We begin with an outline the model as originally laid out by Roy, and follow the notation given in \citet{Heckman2008}. As it was originally formulated, the model considers a number of heterogeneous agents who must choose between two occupations: hunting rabbits and fly fishing. 

Importantly, the skill required to practise these jobs is quite different: rabbits are `slow and stupid,' and so are relatively easy to catch. As a result, there are no particular returns to having great skill at catching rabbits, since skilled trappers will not yield many more rabbits than unskilled trappers. However, the same cannot be said for fly fishing, which is extremely difficult. In this occupation, returns to skills are large: unskilled fishermen will hardly catch anything, but those who have mastered the art can catch a great many fish.

The wage accrued to each activity arises from selling the catch. Fish and rabbits fetch prices $\pi_f$ and $\pi_r$, respectively, and the numbers of each caught by individual $i$ is $F_i$ and $R_i$. Each individual's wage is either given by $w_{fi} = \pi_fF_i$ or $w_{ri}=\pi_rR_i,$ depending on choice of occupation. It is assumed that individuals make their labour supply decisions based only on their wage. $F_i$ and $R_i$ can be considered continuous random variables, so the probability of an agent being indifferent between each occupation is zero.

An important outcome of the model is to explain the {\em selection effect}, or the difference in productivity of individuals in an occupation relative to the population mean, as a result of self-selection. To analyse this effect, suppose that efficiency in each occupation for individual $i$ is normally distributed:
\begin{equation}
 \begin{bmatrix}log{F_i} \\ log{R_i}\end{bmatrix} \sim N(\bmu, \ve{\Sigma}) 
 \label{eq:dist}\tag{1}
\end{equation}
where $\ve{\Sigma}$ is not necessarily diagonal. Roy derived an expression for the average productivity in each sector:
\begin{equation}
 E\left[ log(F_i) | \pi_fF_i \geq \pi_rR_i \right]
   = \mu_f + \frac{\sigma_{ff} - \sigma_{fr}}{\sigma}
     \lambda\left(
       \frac{log(\pi_f) - log(\pi_r) + \mu_f - \mu_r}{\sigma}
       \right)
\label{eq:sr}\tag{2}
\end{equation}
with $\sigma^2$ the variance of individuals' skill ratio, $log(F_i/R_i)$, and $\lambda(\cdot)$ is the inverse Mills ratio, a positive function. 

The expression on the right-hand of \eqref{eq:sr} is the {\em selection effect}, and must be positive for at least one occupation. Specifically, the selection effect is positive for occupations with high skill variance. Further, whether there is positive selection into occupations with {\em lower} variance depends on the covariance between skills ($\sigma_{fr}$ in this example.)

The key challenge arising out of empirical implementations of this model is identification. If the profit maximization and log-normality assumptions can be maintained, and if wages are observable in each sector, then the model is identified. However, the Roy model is frequently used to analyze the labour supply decision, where wages for the household sector are {\em not} observable. In this case, variations {\em across } markets, or variations {\em within } markets (ie across individuals) are used to identify the model.

Three applications of the Roy model are commonly cited. The first is labour supply, where a household sector (with unobserved wages) is added, and labour supply considered a decision to participate in the non-household sector. The second is education: the labour supply decision is the choice between the `high school' and `university' sectors. Profit maximization decisions can assumed based on the cost of further education, as well as the income streams arising out of higher levels of human capital. Finally, many papers use Roy models to model occupational choice, where the `sectors' are occupations derived from census or other survey data, where profit maximization depends on cost of entry (education and certification), utility (or disamentity) of the work, and the expected labour revenue.

\section{Emprical Approach}
The occupational wage distribution has clearly changed over the period 1980-2010. If the polarisation hypothesis is correct, then both the number of full-time workers and their wage distribution will depend on qualitative properties of the occupation. In particular, it is expected that jobs which can be automated (that is, jobs corresponding to a high `automation' score and a low `face-to-face' score) will decrease in number and wage, across the income distribution.

There are a multitude of factors which affect the wage distribution. Analysis by Mincer suggests that wages are determined on the basis of skills alone.

\subsection{Oaxaca-Blinder Decomposition}

\citet{Oaxaca1973} considered the problem of identifying the impact of discrimination against different groups on wage differentials. He proposed an empirical approach that decomposed the influence of human capital variables on wages, from the difference that arise purely from membership in a particular group. Formally, the question is the degree to which the mean difference in log wages 
$$ R = E(\ln y_m) - E(\ln y_f) $$ 
can be explained by a matrix of human capital and experience covariates $\ve{X}\in\mathcal{X}\subset\R^K$, and what is left to be explained by sex. Although Oaxaca was interested in sex discrimination, rather than the changing wage distribution of occupations, the method we consider in this paper is an extension of his original decomposition.

To determine the influence of sex on the mean of the wage distribution, he considered two separate regression models, one for each sex, using data from $n_{m,i}$ males and $n_{f,i}$ females. Each vector includes demographic and human capital variables such as years of education, work experience and age.
\begin{align}
  \ln y_{m,i} &= \ve{x}_{m,i}'\vbeta_m + \epsilon_{m,i} \label{eq:om} \\
  \ln y_{f,i} &= \ve{x}_{f,i}'\vbeta_f + \epsilon_{f,i} \label{eq:of}
\end{align}

Oaxaca's goal was to determine whether there was evidence for sex discrimination, which could not otherwise be explained by the other sociodemographic variables. In other words, to what degree do \eqref{eq:om} and \eqref{eq:of} explain differences in the observed wage distribution, and how much variation remains unexplained? To achieve this, he considered the identity,
\begin{align}
  R &= E[\ln y_{m}] - E[\ln y_{f}] \\
  &=  E[X_m]'\vbeta_m -  E[X_f]'\vbeta_f \notag \\
  &= \underbrace{E[X_m]'(\vbeta_m - \vbeta_f)}_{\text{unexplained by H.C.}} + \underbrace{(E[X_m]'-E[Y_m]')\vbeta_f}_{\text{explained by H.C.}}. \label{eq:odecomp}
\end{align}
This decomposition implies that the difference between the expected log wage between men and women can be explained partially by human capital factors (the `endowments effect' arising from the fact that the distribution of human capital is not independent of sex), and other factors. The parameters in \eqref{eq:odecomp} are computed at their means to determine the difference $E[X_m]'(\vbeta_m - \vbeta_f)$ attributable to discrimination---that is, the proportion of the variance due to group membership---rather than human capital factors. 
%TODO: cite Stata journal

\subsection{Unconditional Quantile Regression}
One major shortcoming of the Oaxaca-Blinder decomposition is that only the conditional means of a wage distribution, $E(Y|X)$, and its counterfactual can be compared. Recall that, in the Roy model described above, changes in the profitability of any occupation should result in the more efficient individuals self-selecting out of an occupation. The mean of a wage distribution is a poor instrument for observing this phenomenon: rather, any polarisation effect will be observed in the overall {\em distribution} of wages, $F_Y$. Ideally, we would like to compute a decomposition similar to \eqref{eq:odecomp}, but which decomposes changes in the $\alpha$th quantile of the wage distribution, $q_\tau(F_Y)$. Such a decomposition was considered by \citet{Firpo2011}; it is their technique, as described in \citet{Firpo2009}, that we apply here.

Suppose that the wage of individual $i$ is observed in two periods, $0$ and $1$. Under the hypothesis of wage polarisation, we will assume that individuals are paid under two distinct wage structures: the pre-polarisation wage structure, $F_{Y_0}$ (when $T=1$) and the post-polarisation wage structure, $F_{Y_1}$ (when $T=0$). We wish to compute an overall change $R^\alpha$ in the quantile statistic, attributable to changes in work force composition $R^\alpha_X$ and changes in the wage structure, $R^\alpha_S$:
\begin{align}
  R^\alpha &= q_\tau(F_{Y_1|T=1}) - q_\tau(F_{Y_0|T=0}) \notag \\
  &= \underbrace{q_\tau(F_{Y_1|T=1}) -  q_\tau(F_{Y_0|T=1})}_{R^\alpha_S} + \underbrace{q_\tau(F_{Y_0|T=1}) q_\tau(F_{Y_0|T=0})}_{R^\alpha_X} \label{eq:decomp}
\end{align}Notice that this decomposition depends crucially on a hypothetical counterfactual distribution, $F_{Y_0|T=1}$, where the workers of period $1$ are paid according to the wage structure of period $0$. Although such a distribution cannot be directly observed, \citet{Firpo2011} show that a consistent estimator of $F_{Y_0|T=1}$ can be found by re-weighting $F_{Y_0}$ to have the same distribution as $F_{Y_1}$.

\citet{Firpo2009} demonstrate that the decomposition described in \eqref{eq:decomp} can be performed using an OLS regression on the recentered influence function of the distributional statistic. This function is the usual influence function used in the analysis of robust estimators, `recentered' by adding the value of the distributional statistic. In the case of the quantile function $q_\tau$, the RIF is given by,
\begin{align*}
  RIF(y; q_\tau) &= q_\tau + IF(y; q_\tau) \\
  &= q_\tau + \frac{q_\tau - \mathbf{1}\{y \leq q_\tau\}}{f_Y(q_\tau)}.
\end{align*}
Then the estimated coefficient $\gamma^{q_\tau}_t$ of a regression of $RIF(y_t; q_\tau)$ on $X$ is
\begin{align*} 
\gamma^{q_\tau}_t &= (E[X \cdot X' | T = t])^{-1} \cdot E[RIF(y_t; q_\tau) \cdot X | T = t]
\intertext{\citet{Firpo2011} show that the distributional statistics themselves can be written as expectations of the conditional RIF, since the expected value of the influence function is zero by definition.}
q_\tau(F_t) &= E_X[E[RIF(y_t; q_\tau) | X=x]] \\
& = E[X|T=t] \cdot \gamma^{q_\tau}_t
\intertext{And thus we can write \eqref{eq:decomp} in a similar form as \eqref{eq:odecomp},}
R^\alpha &= \underbrace{E[X|T=t] \cdot (\gamma^{q_\tau}_1 - \gamma^{q_\tau}_0)}_{R^\alpha_S} + \underbrace{(E[X|T=1] - E[X|T=0]) \cdot \gamma^{q_\tau}_0}_{R^\alpha_X}.
\end{align*}
However, this decomposition is based on a linear specification of the model, where the local approximation at the mean may not be appropriate, particularly for larger changes in covariates.  In particular, if the relationship between $Y$ and $X$ is nonlinear, then shifts in the distribution of $X$ may result in different estimates for $\gamma^{q_\tau}_t$ even if $Y$ is invariant. \citet{Firpo2011} suggest a re-weighting procedure to deal with this problem. They define a re-weighting function, $\Psi(X)$,
\begin{equation}
  \label{eq:wt}
  \Psi(X) = \frac{\Pr(T=1|X)/\Pr(T=1}{\Pr(T=0|X)/\Pr(T=0)},
\end{equation}
that re-weights the data in period $0$ to match the distribution of covariates observed in period $1$.

Using re-weighted data, we can estimate the means of the counterfactual distribution, $\bar{X}=\sum_{i|T=0}\hat{\Psi}(X_i) \cdot X_i$, and the coefficients $\hat{\gamma}_{01}^{q_\tau}$ by regressing $RIF(Y_0;q_\tau)$ with the new sample weights. We then rewrite the decomposition \eqref{eq:decomp} as the sum of two separate Oaxaca-Blinder decompositions. The first term, the wage structure effect, is decomposed into a composition effect $\hat{R}^{q_\tau}_{S,p}$ and specification error, $\hat{R}^{q_\tau}_{S,e}$. The second gives a similar decomposition for composition effect:
\begin{align}
  \hat{R}^{q_\tau} &= (\hat{R}^{q_\tau}_{S,p} + \hat{R}^{q_\tau}_{S,e}) + (\hat{R}^{q_\tau}_{X,p} + \hat{R}^{q_\tau}_{X,e}) \notag \\
  &= \underbrace{\left( [\bar{X}_{01} - \bar{X}_0 ] \hat{\gamma}^{q_\tau}_{01} +
    \bar{X}_{01}[\hat{\gamma}_{01}^{q_\tau} - \hat{\gamma}_0^{q_\tau}] \right)}_{\hat{R}^{q_\tau}_{S}} +
  \underbrace{\left( \bar{X}_{1}[\hat{\gamma}_{1}^{q_\tau} - \hat{\gamma}_{01}^{q_\tau}] + 
    [\bar{X}_{1} - \bar{X}_{01} ] \hat{\gamma}^{q_\tau}_{01}\right)}_{\hat{R}^{q_\tau}_{X}}.
\end{align}

This decomposition can be performed on income surveys of repeated cross-sections of the same markets over time.

\section{Data}

One step which was skipped over in the informal analysis in the previous chapter was the assignment of occupations into task groups, on the basis of the occupational classification scheme. If task content is to be analyzed rigorously, and in greater detail than a simple three-occupation breakdown, a quantitative view of occupational task content is required. 

The standard classification scheme for occupations used in Australia, ANZSCO, simply lists by name the tasks a particular job title might be required to perform. These tasks are listed in an occupation-specific way, such that they cannot be compared between occupations. For example, under the unit group 2243: {\em Economists}, the required tasks include
\begin{quote}
{\em Analysing interrelationships between economic variables and studying the effects of government fiscal and monetary policies, expenditure, taxation and other budgetary policies on the economy and the community \citep[p.185]{Trewin2006}}
\end{quote}

{\em Statisticians} (unit group 2441) perform tasks that are largely similar to that of economists, even though the underlying theory that motivates their work may be somewhat different. A corresponding task entry for statisticians includes
\begin{quote}
{\em Defining, analysing and solving complex financial and business problems relating to areas such as insurance premiums, annuities, superannuation funds, pensions and dividends \citep[p.181]{Trewin2006}}
\end{quote}
Given the qualitative nature of this classification scheme, there is no obvious way to systematically formalise the similarity between economists and statisticians on the basis of the ANZSCO classification. However, alternative classification schemes do exist which include comparable task classifications.

\subsection{Occupational tasks: O*NET}

The U.S. equivalent to the ANZSCO classification, the O*NET database, includes hundreds of quantitative scales for the level of work activities, knowledge types and abilities for individuals in each of approximately five hundred occupations. The data were constructed using expert surveys, and provide a very rich source of information about the activities that workers in each occupation actually undertake. For example, for the work activity {\em analyze data}, the occupations {\em economist} and {\em surgeon} score highly (6.58/7 and 5.49/7, respectively.) But for the work activity {\em Handle moving objects}, surgeons score 3.62/7, and economists score only 0.54/7.

We have mapped the ANZSCO (and its predecessors, various editions of ASCO and the CCLO) to the O*NET data, and sucessfully constructued a skill measure series for Australian occupational classification schemes. We then apply a transformation step, described by \citet{Firpo2011}, to build composite indexes for `automation,' `offshorability', and so on. These composite indexes provide a dependent variable which, along with levels of capital investment on an industry-by-industry basis, provide a basis by which changes in the occupational wage structure can be analyzed.

\subsection{Survey of Income and Housing, 1981-2010}

As in Chapter~2, we use microdata from the ABS Survey of Income and Housing (SIH) data. A detailed discussion of data issues and the processing steps performed on these data files is provided in the Data Appendix.

The survey data are weighted so as to be a representative sample of the Australian population. Weights are given in survey data as `person weights', an inverse selection probability scaled by the current resident population. In order to apply the weighting function in \eqref{eq:wt}, we convert the `person weight' to a selection probability, scale this probability by the weighting function, and then invert this number to give an inverse selection probability. Inverse selection probabilities are then treated as analytic weights, so that the weighted least squares estimator described in Cameron \& Trivedi (2005) can be used.

%Conducting this research for the Australian work force has presented many challenges, particularly when attempting to obtain appropriate data. Unlike the United States, where detailed occupational data appears to be readily available to researchers, we have not been able to obtain survey data for occupations at the four-digit level, which has meant that, when mapping between Australian classification schemes and O*NET, we have had to dramatically reduce the fidelity of our dataset. In general, occupation variables have only been available at the one- and two-digit levels. Unfortunately, comparisons at the two-digit level cannot be made, because during our period of interest of 1981-2010 the ABS has used four different occupational classification schemes. Regrettably, there is no satisfactory way to map between these schemes in a way that is completely comparable, so comparisons must be performed at a higher level of aggregation. For the second part of this study, we are investigating the use of census data instead, for which it may be easier to obtain four-digit data. 

%The decision to use census data was particularly difficult, because this new data brings with it new challenges. The key advantage of the SIH is that the survey is administered by expert interviewers, who are trained to ensure that the income reported by each respondent fits the survey criteria. The resulting income series is of high quality, and is also provided as a continuous variable, so that detail quantile measurements can be made. In the census, respondents do not provide their actual income; instead, income levels are self-reported in binned intervals. Not only does this reduce the accuracy of any analysis performed using census data, but it also necessitates more complicated estimators for changes in the occupational wage structure.

\section{Results and discussion}

%%% Local Variables: 
%%% mode: latex
%%% TeX-master: "paper"
%%% End: 
