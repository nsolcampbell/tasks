\chapter{Empirical Work}\label{ch:4}

\section{Data}

To test the theory of changes in the occupational wage profiles outlined above, we require survey data on real wages, as well as detailed measures of the tasks performed by participants of each occupation. For this analysis, we obtained microdata for the Survey of Income and Housing (SIH) for 1981/82, 2000/01 and 2011/12, as well as measures contained in O*NET database, published by the US Department of Labor. Details of both the task measures and SIH are discussed in detail in Sections~\ref{sec:SIH}~and~\ref{sec:onet} of the data appendix. We shall therefore only briefly review the salient features of the data sources as they relate to this analysis. For further details, refer to Appendix~\ref{app:data}.

\subsection{Survey of Income and Housing}

To bring the SBTC model to the data, we employ the Survey of Income and Housing, a hierarchical clustered household survey conducted by the ABS every 2-3 years since 1995, and also for the fiscal years 1985-6 and 1981-2. The survey provides detailed  information about respondents' labor and non-labor income sources, as well as data on age, educational attainment, hours worked and industry and occupation. For the surveys conducted between 2000 and 2010, as well as the 1981-2 survey, the data include detailed occupational data, which will become important later. The other surveys include occupation only at the one-digit level. We obtain survey micro-data as confidentialized unit record files (CURFs).

To facilitate inter-temporal comparisons, we must eliminate effects which arise as a result of mechanical, demographic shifts. Between 1982 and 2010, the number of women in the work force has increased dramatically, and the same period has seen an evolution of the educational and age composition of the work force, and the rate of casual and part-time employment has increased. Following \citet{Acemoglu2011}, we therefore include only full-time workers for whom labor forms the primary source of income. We further composition-adjust each survey to match 2010 demographics by linearly scaling the survey selection weights for each age group/sex/educational group cell. All computations in this study treat these adjusted weights as inverse selection probabilities.

Repeated cross-sectional measures of the income distribution for full-time salaried workers in Australia are computed using data from the SIH. Consistent with previous work, we consider only the subset of respondents who report working full-time, and whose primary source of income are either employer wages and salaries, or who receive income from an unincorporated business. In order to compare the `market value' of skills, we record employee take-home wages (or revenue from an unincorporated business, after tax), including any additional payments such as entitlements, tips, and bonuses. Revenue from government payments, investments, and so on are ignored. Real incomes are nominal incomes deflated by the average CPI for the four quarters of the fiscal year spanning the survey.

Changes in the occupational coding schemes pose a challenge. In each of the three surveys considered, the occupational coding schemes are different, and cannot be compared directly. In the 1981/82 survey, occupations are recorded using the 1976 Census Classification and Classified List of Occupations (CCLO) codes. Occupations in the 2000/01 SIH are coded using the 1996 Australian Standard Classification of Occupations (ASCO), second edition. And the 2011/12 survey is encoded using the 2006 Australian and New Zealand Standard Classification of Occupations. (A more detailed discussion of the coding systems employed can be found in Appendix Section~\ref{sec:occcoding}). 

For the purposes of this analysis, the income distribution within occupations (or groups of occupations) of different time periods must be comparable. But in the absence of a classification scheme that permits comparison between periods, it is not possible to analyze changes in subsets of the wage distribution. This challenge is not unique to Australian data: occupational coding systems have changed several times in the post-war era in the United States, for example (see \cite{Autor2012,Meyer2005}).

To facilitate comparison, hybrid classification schemes that merged occupations into comparable groups were developed, and are described in detail in the Appendix. There are two such schemes: COMBINEDI, comprising 29 hybrid occupations, for comparing the 1981/82 survey with 2011/12, and COMBINEDII, with 28 hybrid occupations, for 2000/01 and 2011/12 (see Tables~\ref{tab:combined1} and~\ref{tab:combined2}). \citet{Firpo2011} employ a similiar number of hybrid groups (40) in their study of occupational wage changes in the United States between 1988 and 2003. These `consistent' classification schemes can then be linked to occupational task measures, and compared across time periods.

\subsection{Occupational Task Measures}

In order to determine whether specific properties of jobs are associated with changes in the occupational wage profiles, quantiative measures of these properties are required. Unfortunately, although detailed occupational classifications are available for Australian data, these taxonomies do not make available quantitative measures of job attributes.\footnote{Some job information, including tasks and knowledge requirements are available in the ANZSCO and ASCO. Although some quantitative studies have successfully exploited these requirements \citep[e.g.]{Barnes2002}, they are not given in a form that can be readily used for quantitative analysis: see Section~\ref{sec:occclassify} for a discussion.} The lack of quantitative data for Australian jobs need not be a limitation, however. \citet{Goos2009} map occupations for Europe and the U.K. to the U.S. occupational classification scheme in order to exploit O*NET, a comprehensive database of occupational activities, knowledge, job attributes, and working conditions produced by the U.S. Department of Labor. For this analysis, we construct a similar mapping, between the ANZSCO and O*NET at the unit group (four digit) level. We only briefly discuss the procedure for deriving task measures here; a detailed discussion can be found in appendix Section~\ref{sec:onet}.

If the routinsation hypothesis is true, then we expect to see a relationship between the `offshorability' or `routineness' of a job, and its occupational wage distribution. We thus require indexes for these characteristics for each hybrid occupational group, defined above. One problem with the O*NET database is simply its massive size: it contains hundreds of measures, and dozens of different kinds of scales. \citet{Jensen2010} and \citet{Firpo2011} adopt the approach of combining several O*NET indexes to create an aggregate, and we employ Firpo et al's formula for five separate indexes. Three indexes are used as proxies for `offshoreability': {\em information content}, {\em no on-site work} and {\em no face-to-face contact}. To measure `routinization,' we have two indexes: {\em automation/routinization} and {\em no decision-making}.\footnote{These scales are not completely independent. See Appendix Section~\ref{sec:onet} for a discussion.}

\section{The Canonical Model}

\begin{figure}
  \centering
  \includegraphics[width=\textwidth]{../figure/quantile_mf.pdf}
  \caption{Change in weekly wage by percentile, 1981-2010, Males and Females. Full-time workers whose main sources of income are wages and salaries are shown. Notice that real wage growth has been non-monotone for males in lower percentiles. Source: Survey of Income and Housing.}
  \label{fig:banana}
\end{figure}

\subsection{Results}

If SBTC explained the widening of the income distribution, we would expect to observe the premium accruing to `skilled' labor increasing with time. Figure~\ref{fig:banana} shows the composition-adjusted changes in log real wage by percentile, for males and females, between 1981-82 and 2009-10. If the 1981-82 income percentile can be considered a proxy for skill, then it is apparent that, over this period, wages more grew for high-skill individuals much faster than for low-skill individuals. It would therefore be expected that the premium accruing to higher educational attainment would show a similar trend.

In the United States, at least, the wage premium earned by tertiary-educated labor fell in the 1970s, but has risen each decade since then \citep{Acemoglu2011}. \citet{Katz1992} employs a similar empirical model which explains the rise of the skill premium in the United States in the post-war era. In Australia, however, a corresponding growth in the premium for tertiary qualifications has not been observed. Figure~\ref{fig:wagepremium} shows the log skill premium for Australia and the United States between 1982 and 2008. Rather than any fundamental differences in the nature of the demand for skills, \citet{Coelli2009} attributes this difference in Australian workers to differences in the nature of Australian educational qualifications. In Australia, University degrees are available to a wider range of candidates and for a wider range of disciplines than those who would traditionally have undertaken university studies in the United States. As a result, tertiary educational attainment may be a poor proxy for `skilled' work in Australia.
\begin{figure}
  \centering
  \includegraphics[width=\textwidth]{../figure/ed_premium_time_two.pdf}
  \caption{University/non-university log wage premium, Australia and the United States. The figures show the difference between the mean log weekly income for workers who have attained a bachelor degree or higher, and the mean weekly income of other workers. Only full-time workers whose main sources of income are wages and salaries are included, and survey data have been composition adjusted for sex, age group, (and for the United States, race). Source: for Australia, ABS Survey of Income and Housing, and for the United States, \citet{Acemoglu2011}.}
  \label{fig:wagepremium}
\end{figure}

The SBTC model also claims that, even if technology exhibits skill bias, wages for all skill groups should increase monotonically. Figure~\ref{fig:changetime} plots the cumulative change over time for three wage percentiles, the 5th, 95th, and the median. Over the period 1981-82 to 2009-10, although wages at the top percentiles increased steadily, the same is not true for the lower percentiles. Indeed, for all of the 1990s and much of the 2000s, cumulative real income growth from 1981-82 was negative for many workers.
\begin{figure}
  \centering
  \includegraphics[width=\textwidth]{../figure/wage_change_time.pdf}
  \caption{Cumulative log change in real weekly earnings, 5th, 50th and 95th percentiles, 1982-2010. Full-time workers whose main sources of income are wages and salaries are shown. Notice that real wage growth has been non-monotone for males in lower percentiles. Source: Survey of Income and Housing.}
  \label{fig:changetime}
\end{figure}

\subsection{Discussion}

That the income distribution is widening, but the skill premium is {\em not} driving the change, suggests one of at least two interpretations. We have already discussed the fact that educational attainment may be a poor indicator of skill for the Australian labor market. A second, more nuanced explanation was given by \citet{Levy2003}. Technological change may not be complementary to all types of labor; it may replace many types of labor entirely.

\section{The `Disappearing Middle'}

\subsection{Model}

To test this pattern for Australian data, we can augment \eqref{eq:prod} by introducing a third type of labor, $M$, to represent work which requires mid-level skill and low levels of physical activity, representing `routine' or `middling' work. We also introduce computer capital, $C$, as a substitute in production for medium-skilled labor, and a complement in production for high-skilled workers:
\begin{equation}  \label{eq:prod2}
Y = \left[
  \left(A_LL \right)^\frac{\sigma-1}{\sigma}
  +
  \left(A_MM + C\right)^\frac{\sigma-1}{\sigma}
  +
  \left((A_HH)^\mu + C^\mu\right)^\frac{\sigma-1}{\mu\sigma}
  \right]^\frac{1}{\sigma-1}.
\end{equation}
\citet{Michaels2010} use a formulation similar to \eqref{eq:prod2} to show that, if ICT investment $C$ increases exogenously, the wage share for high-skill workers should increase, but decrease for low-skill workers. Likewise, the wage premium for high-skilled workers should rise with increasing ICT investment, and fall for medium-skilled workers.\footnote{Following \citet{Michaels2010}, we focus on the wage {\em share}, and not the absolute wage. Although wages for high-skilled and low-skilled workers should increase with increased investment, the comparative static predictions for medium-skilled workers are indeterminate. Michaels et al. prove that the comparative static predictions for the wage share, however, are unambiguous.} To test these predictions, \citet{Michaels2010} specify a simple translog flexible functional form to test the impact of ICT investment on the wage share for type of labor $S\in\{H,M,L\}$, estimated for broad industry groups across eleven countries, using educational attainment as a proxy for skill. The authors find support for the claim that ICT investment is associated with a decrease in the demand for middle-skilled labor. 

Adapting their specification for Australia gives the empirical model shown below. In this model, $SHARE^S$, computed as ${\sum_k W^S_k/\sum_{s,j}W^s_j}$ is the wage bill share for the labor category $S$, $C$ is ICT capital, $K$ is non-ICT capital, and $Q_i$ is value added by industry $i$. 
\begin{equation} \label{eq:translog}
\Delta SHARE^S = \alpha_{CS}\log(C/Q)_{it} + \alpha_{KS}\log(K/Q)_{it} + \alpha_{QS}\log(Q)_{it}.
\end{equation}
As Michaels {\it et al.} point out, the polarization hypothesis is consistent with $\alpha_{MS}<0$ and $\alpha_{HS}>0$.

With the results from the previous section in mind, to adapt this specification for Australia requires an alternative yard-stick for `skill.' Following \citet{Levy2003}, we partition occupations according to the tasks they involve, according to the occupational classification coded in the SIH. For the purposes of this very simple and informal model, we divide occupations into three categories: `non-routine manual' (low-skilled), `routine' (middle-skilled), and `non-routine cognitive' (high-skilled.) Capital series were derived from national accounting data. Our data include two different measures of ICT capital: {\em software}, and {\em electrical and electronic equipment}. Software includes both commercial off-the-shelf packages, as well as custom-built line-of-business programs, whereas the second variable includes telecommunications equipment and other electronic machinery. To smooth out variation in the data, the  period 1996-2010 was divided into two seven-year periods.

\subsection{Results}

The results from estimating \eqref{eq:prod2}, given in Table~\ref{tbl:reg}, lend mixed support for the polarization hypothesis. While estimates for $\alpha_{MS}<0$ and $\alpha_{HS}>0$ have the expected sign, they are not significant when estimated with all the parameters specified in $\eqref{eq:translog}$. However, with just electrical and electronic equipment included in regression, $\alpha_{MS}<0$ is negative and significant at the 5\% level. Column (4) of Table~\ref{tbl:reg} suggests that, over a seven-year period, a 10\% increase in electrical and electronic equipment capital is associated with a decrease in the wage share of middle-skilled workers of around 0.2, whereas it is associated with a relative increase in the wage share of high-skill workers versus low-skilled workers.

The sign of coefficient estimates for the {\em software} variable are opposites in all estimates. This suggests that software capital may in fact be a complement to medium-skilled labor. Since {\em equipment} includes telecommunications infrastructure, one interpretation is that {\em outsourcing}, rather than a direct application of labor-saving capital, is responsible for the decline in middle-skill labor.

\begin{figure}
  \centering
\includegraphics[width=0.8\textwidth]{../figure/wage_share_equipment_skill.pdf}
  \caption{Change in wage share against change in log ICT electrical and electronic equipment capital ratio, by industry, Australia, 1996-2010.
    Fitted line comuted using LOESS regression and 95\% confidence interval. See note for Table~\ref{tbl:reg} for more details.
  }
  \label{fig:equip}
\end{figure}

\begin{figure}
  \centering
  \includegraphics[width=0.8\textwidth]{../figure/wage_share_software_skill.pdf}
  \caption{Change in wage share against change in log ICT software capital ratio, by industry, Australia, 1996-2010. Fitted line comuted using LOESS regression and 95\% confidence interval.
    See note for Table~\ref{tbl:reg} for more details.
  }
  \label{fig:soft}
\end{figure}

\begin{figure}
  \centering
  \includegraphics[width=0.8\textwidth]{../figure/wage_share_peripherals_skill.pdf}
  \caption{Change in wage share against change in log ICT computers and peripherals capital ratio, by industry, Australia, 1996-2010. Fitted line comuted using LOESS regression and 95\% confidence interval.
    See note for Table~\ref{tbl:reg} for more details.
  }
  \label{fig:periph}
\end{figure}

\documentclass{article}
\usepackage{dcolumn}
\usepackage{rotating}

\begin{document}

% Table created by stargazer v.4.0 by Marek Hlavac, Harvard University. E-mail: hlavac at fas.harvard.edu
% Date and time: Mon, Nov 04, 2013 - 13:11:20
% Requires LaTeX packages: dcolumn rotating 
\begin{tabular}{@{\extracolsep{0.5pt}}lD{.}{.}{-3} D{.}{.}{-3} D{.}{.}{-3} D{.}{.}{-3} D{.}{.}{-3} D{.}{.}{-3} D{.}{.}{-3} D{.}{.}{-3} D{.}{.}{-3} } 
\\[-1.8ex]\hline 
\hline \\[-1.8ex] 
 & \multicolumn{9}{c}{\textit{Dependent variable:}} \\ 
\cline{2-10} 
\\[-1.8ex] & \multicolumn{3}{c}{$\Delta SHARE^H$} & \multicolumn{3}{c}{$\Delta SHARE^M$} & \multicolumn{3}{c}{$\Delta SHARE^L$} \\ 
\\[-1.8ex] & \multicolumn{1}{c}{(1)} & \multicolumn{1}{c}{(2)} & \multicolumn{1}{c}{(3)} & \multicolumn{1}{c}{(4)} & \multicolumn{1}{c}{(5)} & \multicolumn{1}{c}{(6)} & \multicolumn{1}{c}{(7)} & \multicolumn{1}{c}{(8)} & \multicolumn{1}{c}{(9)}\\ 
\hline \\[-1.8ex] 
 $\Delta$ {\em ICT capital} & 0.045 &  &  & 0.102 &  &  & -0.147 &  &  \\ 
  & (0.147) &  &  & (0.198) &  &  & (0.127) &  &  \\ 
  & & & & & & & & & \\ 
 $\Delta$ {\em Non-ICT capital} & 0.236 &  &  & -0.546^{*} &  &  & 0.310 &  &  \\ 
  & (0.190) &  &  & (0.256) &  &  & (0.163) &  &  \\ 
  & & & & & & & & & \\ 
 $\Delta$ {\em equipment} &  & 0.143 &  &  & -0.335^{*} &  &  & 0.191^{*} &  \\ 
  &  & (0.103) &  &  & (0.136) &  &  & (0.089) &  \\ 
  & & & & & & & & & \\ 
 $\Delta$ {\em software} &  &  & -0.049 &  &  & 0.174^{*} &  &  & -0.125^{*} \\ 
  &  &  & (0.059) &  &  & (0.078) &  &  & (0.048) \\ 
  & & & & & & & & & \\ 
 $\Delta$ {\em value added} & 0.178 & 0.064 & 0.030 & -0.131 & 0.081 & 0.190 & -0.047 & -0.145 & -0.220 \\ 
  & (0.172) & (0.142) & (0.147) & (0.232) & (0.187) & (0.193) & (0.148) & (0.122) & (0.120) \\ 
  & & & & & & & & & \\ 
\hline \\[-1.8ex] 
Observations & \multicolumn{1}{c}{32} & \multicolumn{1}{c}{32} & \multicolumn{1}{c}{32} & \multicolumn{1}{c}{32} & \multicolumn{1}{c}{32} & \multicolumn{1}{c}{32} & \multicolumn{1}{c}{32} & \multicolumn{1}{c}{32} & \multicolumn{1}{c}{32} \\ 
R$^{2}$ & \multicolumn{1}{c}{0.063} & \multicolumn{1}{c}{0.067} & \multicolumn{1}{c}{0.028} & \multicolumn{1}{c}{0.149} & \multicolumn{1}{c}{0.179} & \multicolumn{1}{c}{0.155} & \multicolumn{1}{c}{0.179} & \multicolumn{1}{c}{0.179} & \multicolumn{1}{c}{0.227} \\ 
Adjusted R$^{2}$ & \multicolumn{1}{c}{-0.037} & \multicolumn{1}{c}{0.002} & \multicolumn{1}{c}{-0.039} & \multicolumn{1}{c}{0.058} & \multicolumn{1}{c}{0.122} & \multicolumn{1}{c}{0.097} & \multicolumn{1}{c}{0.091} & \multicolumn{1}{c}{0.123} & \multicolumn{1}{c}{0.174} \\ 
\hline 
\hline \\[-1.8ex] 
\textit{Note:}  & \multicolumn{9}{r}{*p<0.05; **p<0.01} \\ 
\normalsize 
\end{tabular} 
\end{document}



These results should be interpreted with caution. Since there is no obvious natural experiment, and nor is there a clear instrument for ICT expenditure, this relationship should be interpreted simply as a correlation. Furthermore, it is unlikely that the level of ICT capital can be considered exogenous, since it is a substitute for endogenously-chosen middle-skilled labor. Nonetheless, the preceding analysis supports the more `nuanced' view that occupational tasks, rather than other human capital variables, are important determinants of the evolution of the wage distribution.

\subsection{Discussion}

The evidence given above is only informal, although it is highly suggestive of a process of polarization in the Australian work force, consistent with patterns found in other labor markets. The results discussed so far also strongly suggest the simple SBTC story does not explain the evolution of the wage distribution in Australia. To wit, the notion of a `skill premium' is problematic in that, in this analysis, educational attainment appears to be a poor proxy of an individual's level of `skill.' Secondly, changes in the distribution of earnings as a result of technological change, appear to depend crucially on the nature of the job, rather than the level of skill it requires that workers possess.

% log-normality of income distribution: \cite{Willis2004}

\section{Tasks and Wages}

In the previous two chapters, we have seen that the `canonical' model of skill-biased technical change does a poor job of explaining the evolution of wage inequality in Australia. In particular, while growing inequality the Australian labour market has mirrored that of overseas economies, there is no empirical evidence that this has been driven by a premium paid to `educated' workers, relative to less educated workers. 

The evidence presented in Chapters~2~and~3 lend weight to Goos~\&~Manning's~(\citeyear{Goos2007}) more `nuanced' interpretation of skill-biased technical change. While educational attainment may explain only little between-group inequality, the data seem to suggest an association between occupational affiliation and the widening wage distribution. This explanation suggests that it is specific attributes of these occupations, and not the education required to undertake them, that explains changes in the wage share. Specifically, it is the `middle-skill' or `routine' occupations described by \citet{Levy2003} and \citet{Goos2009} that can be out-sourced by firms or automated by investments in labour-saving capital equipment. Under this hypothesis, specific attributes of these jobs allow them to be replaced or outsourced, shifts firms' demand curves for these types of labour to the left. As a result of an excess of supply over demand, wages in these occupations are bid down, and wages are both compressed and reduced. 

The analysis presented in Chapter~3 relies on a somewhat arbitrary three-way division of occupations, and presents only correlations between the wage share and capital. Further, this statistical correlation cannot establish a causative relationship between the shrinking wage share of middle-income jobs, and a rising capital-output ratio for the industry. Clearly, a more rigorous analysis is required to demonstrate a clear relationship between tangible properties of middle-skilled jobs and falling wages.

In this chapter, we present a more rigorous analysis, using data on occupational task content compiled by the U.S. Department of Labor to determine which occupations are likely candidates for automation and off-shoring. This data, made available as part of the O*NET database, provides measures of the types of tasks specific occupations entail. Adapting a procedure developed by JK as an extension to \citet{Levy2003}, who adapt the US Dictionary of Occupational Titles, the predecessor to O*NET, to compile indexes for `offshorability' and `routinsation.' These indexes provide a quantiative foundation for comparing changes in the wage distribution and occupations at risk of structural change due to the processes of `offshorability' and `routinisation.'

Before analyzing changes over time, we first describe the relationship between task indexes and occupational conditional means in a single cross-section of the data. Figure~\ref{fig:meanocc4dig} plots the relationship between mean full-time wages, as measured in the 2011 Census, and the task measures constructed from O*NET data.\footnote{Census data are used in Figure~\ref{fig:meanocc4dig}, rather than the SIH, because occupational wages are available a greater level of detail: ANZSCO unit groups (four digit), rather than minor groups (two digit). The same chart is replicated using SIH data in Figure~\ref{meanocc2dig}; the patterns that emerge are almost identical.} The data are plotted at the ANZSCO unit group (four digit) level, with a loess regression line, weighted by occupation population, superimposed.

\begin{figure}
  \centering
  \includegraphics[width=\textwidth]{../figure/wages_indexes_4digit.pdf}
  \caption{Mean occupational weekly wage and task measure index values, at ANZSCO unit group (4-digit) level. The vertical dashed line is drawn at the level of the National Minimum Wage, of \$589.30 per week. Census respondents reporting full-time work are shown. The loess regression line is weighted by population; circle areas are proportional to population for each occupation. Notice that, when occupations are reduced to combined groupings, almost identical trends are observed (c.f. Figure~\ref{fig:meanocc2}). Sources: ABS cat 2072.0, O*NET, US Dept of Labor.}
  \label{fig:meanocc4dig}
\end{figure}

Two obvious patterns emerge in Figure~\ref{fig:meanocc4dig}: the information content and decision-making indexes are strongly positively related to conditional mean wages. These relationships are hardly surprising: professional and managerial work, which tends to be relatively highly remunerated, typically involves information processing and a greater degree of decision-making. Similarly, a negative relationship between automation/routinization and conditional mean wages is also evident. As \citet{Goos2009} argue, so-called `lovely' jobs, which are usually relatively well-paid, tend to involve primarily nonroutine activities, whereas lower-paid `middling' jobs tend to involve a greater proportion of repetitive activity. Finally, there does not appear to be a simple relationship between the face-to-face or on-site task indexes.

In order to test the routinization and outsourcing theories of occupational wage change, it is not enough to examine cross sections of the wage distribution at a given point in time. Rather, since our theory posits an increase in wage dispersion as a consequence of technical change, then these changes should be evident over a period of time. That there is a downward-sloping relationship between automation/routinization and conditional wages is insufficient: it must be demonstrated that this relationship is becoming stronger over time.

\section{Wage Profile Changes}\label{sec:direct}

%As shall be discussed below, the key challenges arising out of empirical implementations of model of this type is identification. If the profit maximization and log-normality assumptions can be maintained, and if wages are observable in each sector, then the model is identified. However, when a Roy model is used to analyze a labour supply decision, where wages for the household sector are {\em not} observable, and nor are the talents of individuals, then the parameters in \eqref{eq:sr} are not identified. In this case, variations {\em across } markets, or variations {\em within } markets (ie across individuals) are used to identify the model.

The desired decomposition is a relationship between occupations and their constituent tasks. Roy-type models posit that the wage an individual is paid depends on the skills demanded by that occupation, and the returns to the skills in question. One simple approach to identifying the contribution of each one of a worker's skills to the overall wage, considered by \citet{Firpo2011}, is to adapt the simple linearly additive functional form of \citet{Welch1969}. Welch assumed that an individual's wage is determined linearly by the individual skills that worker possessed.
\begin{assumption}[Linear additivity of returns to skills] \label{ass:linear}
  An individual $i$'s wage in occupation $j$ at time $t$ is set according to the sum of the returns $r_{jk}$ to skills $k$, $k=1,\dots,K$ required for that occupation:
\begin{equation}
  w_{ijt} = \theta_{jt} + \sum_{k=1}^K r_{jkt}S_{ik} + u_{ijt}, \label{eq:linear}
\end{equation}
Where $\theta_{jt}$ is a `base pay' term, and $u_{ijt}\sim i.i.d$ captures idiosyncratic characteristics of each worker. 
\end{assumption}

This is a strong assumption, which enjoys limited empirical support. Linear additivity implies that the labour market in each occupation is free of general equilibrium effects arising from changes in other occupational wage structures.

Notice that in \eqref{eq:linear}, the returns to skill $k$ are particular to occupation $j$. This makes intuitive sense: since each individual is endowed with a particular mix of skills, which may not necessarily be useful in that individual's chosen occupation, there is no reason to expect the returns to certain skills to equilibrate across markets. In Roy's example above, fly fishing skills of any level do not earn a return for workers engaged in rabbit hunting. % literature rejecting skill bundling

\citet{Firpo2011} perform two separate analyses of changes in the occupational wage. The first, outlined below, directly analyses the occupational wage profile as quantiles.

As a first step in the analysis, we directly analyze the relationship between occupational task measures and changes in the aggregate occupational wage profile. Under the maintained assumption that wages are linearly separable, it follows that changes in the occupational returns to a particular skill $r_{jk}$ will be observable in the aggregate wage profile.

Estimating a similar model of wage profiles, \citet{Juhn1993} suggested that, in regressions on occupational wage quantiles, a worker's rank was a good instrument for that worker's ability. Thus, in aggregate, a fixed quantile effect $\lambda^q$ across groups could be interpreted as an aggregate measurement of changing returns to ability. We first estimate changes in the wage quantiles, for each occupation $j$ and each quantile $q$,
\begin{equation} 
  \Delta w_j^q = a_j + b_jw_{j0}^q + \lambda^q + \epsilon^q_j, \label{eq:quantiles}
\end{equation}
where $\lambda^q$ is an estimate of returns-to-skill at each quantile $q$. Under the maintained assumption that the returns to skill at each quantile of the wage distribution is independent of the actual occupation, then the parameters $a_j$ and $b_j$ describe the changes in each occupation over the study period.

The next step in the analysis is to decompose these changes according to the task we are interested in. These task measures, defined below, capture the ability to off-shore or automate an occupation. Applying the first-step regressions defined above, we are now in a position to test the relationship between changes in occupational wage profiles and task indexes:
\begin{align}
  \hat{a}_j &= \gamma_0 + \sum_{h=1}^K\gamma_{jh}TC_{jh} + \mu_j, \label{eq:aeq} \\
  \hat{b}_j &= \delta_0 + \sum_{h=1}^K\delta_{jh}TC_{jh} + \nu_j. \label{eq:beq}
\end{align}

\subsection{Results}

The Roy model outlined above posits that, if the demand for labour of a particular typee is shifting to the left, then two changes in the wage distribution should be visible: both mean wages and wage disperson should decrease. To test for these changes in the occupational wage distribution, we fit the model described in Section~\ref{sec:direct} for two periods: from 1981/82 to 2011/12, using grouping I, and from 2000/01 to 2011/12, using grouping II. Once we have estimates of the change in mean and dispersion of the occupational wage distribution, we regress both of these measures against our task measures. Under the model hypothesized above, we therefore expect to obtain negative coefficient estimates for all five task measures.

Second-stage regression results for the periods 1981/82 to 2011/12 and 2000/01 to 2011/12 are tabulated in Tables~\ref{tab:quantreg1} and~\ref{tab:quantreg2}, respectively. In both tables, models 1---3 represent estimation results for \eqref{eq:aeq}, where the change in mean, $a_j^q$, is the dependent variable, and models 4---6 represent estimation results for regressions on the slope term $b_j^q$, specified in \eqref{eq:beq}. In models (1) and (4), coefficients associated with both outsourcing and routinizaton are entered together; whereas just outsourcing variables feature in models (2) and (5), an routinization variables in (3) and (6). Importantly, the sign and significance of the estimates in both tables are very similar, despite one dataset spanning 30 years, and the other a little over a decade. This suggests that occupational wage changes captured by the model are relatively recent. For the purposes of our discussion here, we will restrict our attention to Table~\ref{tab:quantreg2}, which covers the period 2000/01 to 2011/12. Note that while the sign of the coefficients can be interpreted, since these task measures were compiled from unit-free indexes and then arbitrarily normalized to have a unit range, the scale of the coefficients is aribitrary and has no direct interpretation. In particular, the reader is cautioned against comparing coefficient estimates between indexes; it is not clear that this would be at all meaningful.

\documentclass{article}

\begin{document}

% Table created by stargazer v.4.5.1 by Marek Hlavac, Harvard University. E-mail: hlavac at fas.harvard.edu
% Date and time: Thu, Oct 10, 2013 - 12:36:34
\begin{table}[!htbp] \centering 
  \caption{Intercept and Slope of Change in Wage Quantiles, 1981/2 - 2011/12} 
  \label{} 
\begin{tabular}{@{\extracolsep{5pt}}lcccccc} 
\\[-1.8ex]\hline 
\hline \\[-1.8ex] 
 & \multicolumn{6}{c}{\textit{Dependent variable:}} \\ 
\cline{2-7} 
\\[-1.8ex] & \multicolumn{3}{c}{A (intercepts)} & \multicolumn{3}{c}{B (slopes)} \\ 
\\[-1.8ex] & (1) & (2) & (3) & (4) & (5) & (6)\\ 
\hline \\[-1.8ex] 
 Information.Content & 3.31 & $-$0.73 &  & $-$1.11 & 0.97 &  \\ 
  & (2.09) & (1.56) &  & (0.92) & (0.72) &  \\ 
  & & & & & & \\ 
 Automation.Routinization & $-$6.06 &  & $-$9.38$^{***}$ & 3.11$^{*}$ &  & 5.12$^{***}$ \\ 
  & (4.11) &  & (2.97) & (1.81) &  & (1.31) \\ 
  & & & & & & \\ 
 No.Face.to.Face & $-$0.01 & 1.39 &  & 0.26 & $-$0.48 &  \\ 
  & (2.96) & (2.29) &  & (1.30) & (1.05) &  \\ 
  & & & & & & \\ 
 No.On.Site.Work & 0.66 & 3.09$^{**}$ &  & $-$0.48 & $-$1.74$^{***}$ &  \\ 
  & (1.36) & (1.12) &  & (0.60) & (0.51) &  \\ 
  & & & & & & \\ 
 No.Decision.Making & 7.23$^{**}$ &  & 5.14$^{**}$ & $-$3.73$^{***}$ &  & $-$3.13$^{***}$ \\ 
  & (2.70) &  & (1.96) & (1.19) &  & (0.86) \\ 
  & & & & & & \\ 
 Constant & $-$1.35 & $-$0.97 & 3.46$^{***}$ & 0.39 & 0.19 & $-$1.68$^{***}$ \\ 
  & (1.85) & (1.52) & (1.05) & (0.81) & (0.70) & (0.46) \\ 
  & & & & & & \\ 
\hline \\[-1.8ex] 
Observations & 28 & 28 & 28 & 28 & 28 & 28 \\ 
R$^{2}$ & 0.50 & 0.33 & 0.29 & 0.57 & 0.38 & 0.40 \\ 
Adjusted R$^{2}$ & 0.38 & 0.25 & 0.23 & 0.48 & 0.30 & 0.35 \\ 
Residual Std. Error & 394.00 (df = 22) & 434.54 (df = 24) & 439.21 (df = 25) & 173.29 (df = 22) & 199.84 (df = 24) & 193.06 (df = 25) \\ 
F Statistic & 4.33$^{***}$ (df = 5; 22) & 3.96$^{**}$ (df = 3; 24) & 5.07$^{**}$ (df = 2; 25) & 5.90$^{***}$ (df = 5; 22) & 4.91$^{***}$ (df = 3; 24) & 8.25$^{***}$ (df = 2; 25) \\ 
\hline 
\hline \\[-1.8ex] 
\textit{Note:}  & \multicolumn{6}{r}{$^{*}$p$<$0.1; $^{**}$p$<$0.05; $^{***}$p$<$0.01} \\ 
 & \multicolumn{6}{r}{Occupational grouping #1 used, with 28 occupational groups.} \\ 
\normalsize 
\end{tabular} 
\end{table} 
\end{document}



\documentclass{article}

\begin{document}

% Table created by stargazer v.4.0 by Marek Hlavac, Harvard University. E-mail: hlavac at fas.harvard.edu
% Date and time: Thu, Oct 03, 2013 - 20:58:02
\begin{table}[!htbp] \centering 
  \caption{Intercept and Slope of Change in Wage Quantiles, 2000/01 - 2009/10} 
  \label{} 
\begin{tabular}{@{\extracolsep{5pt}}lcccccc} 
\\[-1.8ex]\hline 
\hline \\[-1.8ex] 
 & \multicolumn{6}{c}{\textit{Dependent variable:}} \\ 
\cline{2-7} 
\\[-1.8ex] & \multicolumn{3}{c}{A (intercepts)} & \multicolumn{3}{c}{B (slopes)} \\ 
\\[-1.8ex] & (1) & (2) & (3) & (4) & (5) & (6)\\ 
\hline \\[-1.8ex] 
 Information.Content & $-$0.47 & $-$0.95$^{**}$ &  & 0.08 & 0.18$^{***}$ &  \\ 
  & (0.48) & (0.37) &  & (0.07) & (0.06) &  \\ 
  & & & & & & \\ 
 Automation.Routinization & $-$0.54 &  & $-$1.24$^{***}$ & 0.07 &  & 0.19$^{***}$ \\ 
  & (0.52) &  & (0.36) & (0.08) &  & (0.06) \\ 
  & & & & & & \\ 
 Face.to.Face & $-$0.02 & $-$0.02 &  & $-$0.01 & 0.01 &  \\ 
  & (0.49) & (0.33) &  & (0.08) & (0.05) &  \\ 
  & & & & & & \\ 
 On.Site.Job & $-$1.00$^{**}$ & $-$1.40$^{***}$ &  & 0.16$^{**}$ & 0.23$^{***}$ &  \\ 
  & (0.41) & (0.33) &  & (0.06) & (0.05) &  \\ 
  & & & & & & \\ 
 Decision.Making & $-$0.74 &  & $-$1.26$^{***}$ & 0.14$^{*}$ &  & 0.22$^{***}$ \\ 
  & (0.46) &  & (0.37) & (0.07) &  & (0.06) \\ 
  & & & & & & \\ 
 Constant & 8.03$^{***}$ & 7.64$^{***}$ & 7.87$^{***}$ & $-$1.24$^{***}$ & $-$1.22$^{***}$ & $-$1.23$^{***}$ \\ 
  & (1.10) & (0.71) & (0.71) & (0.17) & (0.11) & (0.11) \\ 
  & & & & & & \\ 
\hline \\[-1.8ex] 
Observations & 29 & 29 & 29 & 29 & 29 & 29 \\ 
R$^{2}$ & 0.51 & 0.45 & 0.35 & 0.54 & 0.46 & 0.38 \\ 
Adjusted R$^{2}$ & 0.40 & 0.38 & 0.30 & 0.44 & 0.39 & 0.34 \\ 
Residual Std. Error & 15.67 (df = 23) & 15.88 (df = 25) & 16.96 (df = 26) & 2.40 (df = 23) & 2.48 (df = 25) & 2.60 (df = 26) \\ 
F Statistic & 4.72$^{***}$ (df = 5; 23) & 6.80$^{***}$ (df = 3; 25) & 6.90$^{***}$ (df = 2; 26) & 5.32$^{***}$ (df = 5; 23) & 7.08$^{***}$ (df = 3; 25) & 8.11$^{***}$ (df = 2; 26) \\ 
\hline 
\hline \\[-1.8ex] 
\textit{Note:}  & \multicolumn{6}{r}{$^{*}$p$<$0.1; $^{**}$p$<$0.05; $^{***}$p$<$0.01} \\ 
 & \multicolumn{6}{r}{Occupational grouping #2 used, with 29 occupational groups.} \\ 
\normalsize 
\end{tabular} 
\end{table} 
\end{document}



The evidence for the routinization and outsourcing theories presented in Table~\ref{tab:quantreg1} is somewhat mixed. As expected, a higher level of routinzation in an occupation is associated with a decrease in wages, across all quantiles of the wage distribution. However, automation is also associated with {\em greater} wage dispersion, not less. Consistent with the theory, the slope terms for `no on-site work' and `no decision-making' are both signficiantly negative, so that offsite work and a lack of decision-making are both associated with a decreased dispersion of wages. However, contrary to the theory, estimates for changes in the mean wage are significantly {\em positive}, the opposite of the sign predicted by the theory. Similarly, the evidence for `information content' is conflicting: although we expect both the change in mean and dispersion to be negative, the change in dispersion is in fact significantly positive.

The results in Table~\ref{tab:quantreg1} stands in contrast to \citet{Firpo2011}, who found that, in the United States between 1988 and 2002, three out of the five indexes given here are associated with negative changes in both the mean and dispersion of wages. Indeed the mixed results discussed above suggests a number of possible explanations. First, it is possible that the proposed Roy model is inadequate, and simply does not explain changes in occupational wage profiles. Second, changes in the wage profile may not be uniform across the earnings spectrum, as \eqref{eq:quantiles} assumes. Finally, it could be that these results are simply an artefact of the occupational mapping or aggregation scheme employed, or structural differences between the United States and Australian labour market.

One major difference with US labour market is the presence of a sizeable minimum wage in Australia. In Figure~\ref{fig:meanocc4dig}, notice that the 2011 National Minimum Wage of \$622.30 per week, illustrated by the dashed line, is quite close to the conditional mean wage of some occupations. It is therefore possible that, at some wage levels, changes in the mean or dispersion of wages have very little to do with the properties of the job, but instead are due to institutional factors. Proximity of the wage distribution to the level of the level of the minimum wage suggests that the presence of non-linearities in the relationship between tasks and wages could be important.\footnote{The results presented in Tables~\ref{tab:quantreg1}~and~\ref{tab:quantreg2} are robust to the exclusion of subsets of quantiles. In particular, excluding quantiles close to the minimum wage has negligible effect on parameter estimates.}

\section{Tasks and Wages: A Decomposition Approach}

% ** todo: summarize reweighting and RIF regression methods here


\subsection{Results}

\begin{figure}
  \centering
  \includegraphics[width=\textwidth]{../figure/rif2.pdf}
  \caption{Marginal log wage effect of task measures on log wage quantiles, 1981/82--2011/12 and 2000/01--20011/12, with shaded 95\% confidence intervals. At any given quantile, overlap between the confidence band and the x-axis indicates a lack of statistical significance. The top row shows unconditional quantile regressions against task measures for 1981/82 and 2011/12, and the bottom row, 2000/01 and 2011/12. The vertical axis measures $\partial\ln(w^q)/\partial T_i$, the marginal impact on the log wage of a unit change of the task measure. Notice the similarity between the 2011/12 curves under both coding schemes. This similarity suggests that occupational coding schemes map consistently to the underlying O*NET task measures. Sources: ABS SIH 1981/82, 2000/01, 2011/12; ABS cat. no. 6401.0, 1220.0, 1223.0, 1288.0.; U.S. Dept of Labor.}
  \label{fig:rif}
\end{figure}

Figure~\ref{fig:rif} presents the results of unconditional quantile regressions of task measures against the occuptional wage profile, afer accounting for demographic and human capital variables.\footnote{Eight dummies for potential experience, education, sex and martial status are included.} On the left-hand side, the first classification scheme is illustrated, and on the right, the second. The top row shows the base period (1981/82 and 2000/01, respectively), and the bottom row shows 2011/12 for both classification schemes. The horizontal axis shows the quantiles of the (real) wage distribution, and the vertical axis is measured in log points per scale unit. 95\% confidence intervals are shaded, so that at each quantile, statistical signficiance is indicated by the shaded area not overlapping the horizontal axis.\footnote{Note that, since task measure scale units are essentially arbitrary, it is not meaningful to compare different task measures vertically.}

Figure~\ref{fig:rif} illustrates two important facts. First, notice that the marginal task impact curves differ between periods. Consequently, a unit change in occupational task measures is associated with a different impact on wage quantiles at the start and end of both periods. We will focus on these differences, below. Second, notice that marginal effects for each period and task measure appear to be related to quantiles (the x-axis) in a complex way. At this stage, we can conclude that the association between the wage distribution and task measures is highly nonlinear, and that the simple model estimated in \eqref{} is not rich enough to capture the changes over time in the wage distribution.

% *** describe the cross-sectional patterns in broad brushstrokes here ***

Although their shapes are indicative, the difference between the marginal task effects between periods,
$$\partial\ln(w^q_{T=1})/\partial TC_i-\partial\ln(w^q_{T=0})/\partial TC_i,$$
need to be interpreted with care. Recall from the previous dicusssion that observed changes in the marginal income distribution can occur over time for two reasons. First, a change in the composition of the population of individuals self-selecting into occupations. If, for instance, individuals with a higher degree of human capital were to self-select into occupations with a higher level of a particular task measure, then the observed marginal effect of that task measure would increase. This change corresponds to the $\Delta_X$ term of \eqref{eq:odecomp}. The second component of changes in the marginal effect of task measures is associated with structural changes in the occupational wage structure, denoted $\Delta_S$. In the following section, we now formally de-compose the changes in the occupational wage structure into these two components.

\subsection{Decomposition Results}

\subsection{Conclusions}

%%% Local Variables: 
%%% mode: latex
%%% TeX-master: "paper"
%%% End: 
