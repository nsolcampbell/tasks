\chapter{Empirical Literature}\label{ch:3}

In this chapter, we review empirical evidence for the models discussed in Chapter~\ref{ch:2}. To some extent, the distinction between the theoretical and empirical literture is artificial: models of wage differentials and technology are difficult to separate from the empirical regularities that they describe. Nonetheless, in this section, we discuss four broad classes of studies.

First, we briefly discuss results from demographic studies of the work force. These `model-free' studies are important because they provide the empirical regularities that models are intended to explain. Second, we look at estimates of skill changes, based on classification schemes and other measures. Third, we discuss estimates of neoclassical models of the labor force, in which model parameters are calibrated to mean values derived from survey and aggregate data. Finally, we review some examples of decomposition-type studies, in which non-parametric and semi-parametric evidence for wage-setting models are drawn from empirical wage distributions.

There is a large body of evidence for upskilling and polarization in foreign labor markets; indeed this evidence prompted much of the research into skill-biased technical change. We touch briefly on these studies, but where possible, our focus here is on Australian research. Somewhat surprisingly, while one of the key explanations for SBTC includes globalization and the worldwide proliferation of new technology, the evidence for SBTC in Australia does not align with the US and European experience. Many Australian studies have confirmed a growing demand for skilled labor, as well as an associated growth in its supply, however there is less evidence for SBTC in Australian wage data.

\section{Direct Measures of SBTC}\label{sec:ethnographic}

One way to determine whether technology is skill-biased is to directly analyze the properties and wage distributions of jobs that use new technologies. One advantage of this type of study is the absence of an explicit economic model---so the results are less susceptible to specification biases. To support the SBTC theory, two regularities should be verified in survey data: first, technology adoption should be growing, and that the nature of work changes as a result (the existence of technical change). Second, one expects to see that the impact of this technology falls primarily on the work of those with `high' skills (the existence of skill bias).

Qualitiative research on the nature of computerization in foreign labor markets strongly supports the claim that technology changes workplaces. Evidence from the US Current Population Survey confirms that, during the 1980s and 1990s, there was indeed an increased incidence of computer use in the workplace. Between 1984 and 1997, the data show that the proportion of individuals using computers at work increased from 24\% to 51\% \citep{Friedberg2003}. Furthermore, evidence from the 1990s shows that the introduction of new technology results in a substantial rearranging of work patterns. \citet{Levy1996} studied the application of new technology to automate tasks in a financial services firm. He found that, although technology simplified many of the processes, those that were not automated became more complicated. Similarly, \citet{Autor2002} studied the introduction of digital check imaging in a large bank, and found that, while many `routine' tasks were easily automated, substantial changes occurred in those tasks that could not be performed by machines. \citet{Bresnahan2002} offers evidence that the introduction of computers into workplaces often incurs significant adjustment costs, including re-training, re-organization, and so on.

A limited number of surveys have been conducted in Australia. \citet{Borland2004} analyze a cross-sectional survey of Australian workers, the 1993 ABS Training and Education Experience Survey (TEES). This survey included detail of workers' skills and depth of computer knowledge, as well as interval-censored earnings information. Using interval regression techniques, Borland regressed a number of human capital, experience and job characteristic variables against income, worker characteristics and proxies for skills, as well as a categorical variable concerning computer use and experience. 

Without proxy measures of unobservable ability, the return to computer use in 1993 is estimated at 18 per cent of earnings; however, once controls are included, this effect reduces to about 8 per cent. One problem with this type of study is that, since computer use is associated with high-skilled work, unobserved ability is likely to be correlated with computer use. Despite the inclusion of measures of individual skill and ability, this means that the return to computer use cannot be exactly identified. Nonetheless, this study does strongly suggest the presence of a skill premium in the Australian labor market.

\section{The `College Premium'}

Beginning in the 1980s, a divergence between the rental rates of skilled and unskilled labor began to emerge in the US. \citet{Acemoglu2011} report that the `skill premium' paid to college-educated workers remained relatively steady between 1964 and 1980, oscillating in the range of between 48 and 58 per cent above that of other workers, if other factors are held constant. However, from 1980, this premium increased steadily, far outpacing the growth rates of other types of labor, dramatically increasing wage inequality between income groups. This trend was documented using CPS microdata by \citet{Karoly1992} and \citet{Katz1989}, among others, using reported educational attainment to classify  individuals into groups.\footnote{The US experience was not universal: \citet{Katz1989} found, for example, that Japanese wage differentials had not followed the same pattern.} These studies also identified an intensification of the proportion of workers in the US with tertiary qualifications. To some extent, there were factors unrelated to the work force that could explain this jump in educational attainment: in the 1970s, students who continued college study were exempt from service in Vietnam, and returning veterans were granted scholarships via the G.I. bill \citep{Acemoglu2011}. Nonetheless, the two labor market trends emerging from this literature were a continual `up-skilling' of the workforce since the 1980s, and a steady increase in the `college premium', the average premium paid to workers who had attained college degrees or higher.

\subsection{Upskilling in Australia}\label{sec:upskillingau}

`Skills' in the Australian labor market have been identified in empirical studies using (at least) two methods. The first, as with the US studies outlined above, is to use educational attainment. The second is to use occupational group as a proxy for skill.

In sample surveys, occupations must be coded according to some consistent scheme in order to be comparable across the sample surveyed, and also across time periods. The occupational classification schemes used in Australia, such as the ASCO \citep{Castles1986}, and the ANZSCO \citep{Trewin2006}, typically include a `skill level' ranking for each coded occupation. The ASCO II, defines skill level as `a function of the range and complexity of the set of tasks involved' \citep[p.14]{Castles1986}, and is a combination of the level of education and experience required for classified occupations. The use of these skill metrics have been criticized, but they do provide useful categories for analysis of the skill distribution.\footnote{\citet{Cully1999} argues their use originates from the need to place employees in a social class, rather than analyze the skill intensity of the work. However, it is not clear that these skill levels are any worse than the use of educational attainment as a skill proxy.}

\citet{Cully1999} divides employment data into five broad skill groups, and analyzes changes in the number of jobs in each category between 1986 and 1997.\footnote{These groups are aggregations of the top-level groupings given in the ASCO, and are the same was those later used by \citet{Wooden2000} and \citet{Esposto2012}. The skill groupings are, (1) managers/professionals, (2) associate professionals, (3) skilled vocations, (4) intermediate skills, (5) elementary skills.} He finds that, over this period, there was growth in the high- and low-skilled jobs, but that the number of jobs in the middle categories was stagnant or declined, lending some support to the polarization hypothesis. However, in line with international evidence, the dominant pattern in the data is growth in high-skilled jobs, or `up-skilling.' A similar conclusion is reached by \citet{Wooden2000}, who expands Cully's methodology to also consider growth in terms of hours worked.

A more recent analysis of the skill distribution was undertaken by \citet{Esposto2012}, adopting the aggregation approach taken by \citet{Cully1999}, but using a longer sample and Census data, between 1971 and 2006. By recombining the 1971 census data, which was coded using an obsolete coding scheme, to suit the ASCO II classification, Esposto was able to create a comparable series of the levels of occupations over this 35-year period. Novel in Esposto's analysis is a breakdown of full-time and part-time work, a distinction that has grown in importance over the 2000s.

Esposto found that, between 1971 and 2006, the skill intensity of the Australian labor market increased, as measured by a greater proportion of the work force employed in high-skilled occupations. The greatest growth was observed in the top category, professionals and managers (115.5 per cent), and the two lowest skill categories, intermediate and elementary skills, also grew (54.5 and 22.5 per cent, respectively). The middle category, `skilled vocations' shrank by 7.4\%, suggesting job polarization. Furthermore, Esposto disaggregated part-time and full-time jobs, and found that full-time jobs predominantly experienced upskilling, whereas part-time and casual jobs experienced down-skilling, especially for males.

\subsection{A `Uni Premium' in Australia?}

Somewhat surprisingly, a college premium is not readily apparent in the Australian data. \citet{Barnes2002} analyze household survey data, and use educational attainment as a proxy for skill. They find that, over the 1980s and 1990s, growth in the demand for high-skilled far outpaced that of low-skilled workers. In the 1980s, the demand for skilled employment grew at a rate of 4.7 per cent per year (against 0.5 per cent annually for unskilled unemployment), and in the 1990s, the growth rates were 3 per cent and 0.8 per cent, respectively.

In contrast to the American and European experience, \citet{Barnes2002} find no evidence of a skill premium. Using industry measures as proxies for demand, the authors attempt to decompose the differences between relative demand and supply for each type of labor, and find negligible discrepancies. They conclude that the lack of a college premium  is due to the supply of both types of labor expanding at the same rate as their respective demands, creating no scarcity premium in the labor market.

No college premium was found by \citet{Coelli2009}, who followed a similar procedure to \citet{Katz1992}. the authors employ both income survey microdata and census samples to estimate the premium paid to university graduates, in excess of those without university degrees, between 1981 and 2004. Like \citet{Barnes2002}, Coelli and Wilkins find that, that although a university premium exists, it is not rising, as it is in the United States and Europe.

Coelli suggests a novel explanation for the absence of a rising wage premium for degree-qualified labor. First, they note differences between the Australian and US system of tertiary qualifications. In the US, bachelor degrees take four years to attain, but in Australia, three-year degrees are the norm, suggesting that the attaiment of a three-year degree. They also point out that changes to higher education funding arrangements in the 1990s have broadened the scope of teritary degrees tremendously, so that many degrees now cover skills that would previously have been taught at technical colleges such as TAFE. If the scope of what is taught at universities is expanded to include skills not normally associated with high-skilled work, then a relatively smaller proportion of university graduates will be observed to be in high-skill, high-earning occupations, and the measured college premium will be lower.

\section{Models of SBTC}

Recall from Chapter~\ref{ch:2} (\S\ref{sec:canonical}) that, as the technology coefficient associated with high-skilled workers increases, the canonical model of SBTC predicts a rising premium to be paid to high-skill workers. Indeed, evidence from the United States and Europe support this claim.

\citet{Katz1992} estimate a version of the college premium, as described in \eqref{eq:omega}. To do so, they assume an exponential functional form for the evolution of the technology ratio, $A_H/A_L$, over time,
$$  (A_H/A_L)(t) = A_{0}e^{A_1t}, $$
where $A_0$ and $A_1$ are constants. When substituted into \eqref{eq:omega}, this yields a regression model of the following form:
\begin{equation}\label{eq:regcanonical}
\log \omega_t = \frac{\sigma - 1}{\sigma}\beta_0 + \frac{\sigma-1}{\sigma}\beta_1t - \beta_2\log\left(\frac{H_t}{L_t}\right) + \epsilon_t,
\end{equation}
where $\log(H/L)$ is the log wage share ratio. Using data from 1963 to 1987, they estimate
$$
  \log \omega_t = \kappa + \underset{\scriptsize (0.007)}{0.033} t - \underset{\scriptsize (0.150)}{0.709}\left( \frac{H_t}{L_t} \right),
$$
where $\kappa$ is a constant. This model implies a college premium rising at a rate of approximately 3.3 per cent annually.

As \citet{Acemoglu2011} point out, this model predicts the rise in the college premium over the 1990s reasonably well, although it does under-predict the true level of between-group inequality somewhat from 2002 onwards.

\subsection{SBTC in Australia}

Under the canonical model, the proportion of high-skilled labor employed should increase in the presence of SBTC. Using industry-level data between 1978 and 2000, \citet{DeLaine2001} analyze the changes in shares of skilled and unskilled labor, identified by educational attainment. They find that both the total wage bill and share of employment of skilled workers has increased dramatically, across all industries, in Australia over this period.

They further test whether technology investment or technology use indexes can explain this evolution. To do so, they employ a variety of functional forms, including a CES production function and a flexible (translog) model to estimate changes in the shares of high-skilled labor, as a function of R\&D spending, capital growth and a technology index. 

The manufacturing industry, when entered alone, shows a strong relationship between the share of skilled workers and technological change. However, the authors find that this relationship is weaker for other industries. \Citet{DeLaine2001} find that the relationship strengthens in the 1980s, and posit that this period of extensive microeconomic reform allowed firms greater flexibility to adopt new technologies requiring a more highly skilled work force.

%\section{More Nuanced Approaches}

\section{Occupational Task Measures}

As \citet{DiNardo1997} point out, the assumption that changes in wage premia observed the last two decades of the 20th century are due to technological change may incorrect, or overwrought. Indeed, there are several competing models capable of explaining this trend (\S\ref{sec:risingpremia}). In the literature of the 1990s, the argument for SBTC rested upon two pieces of evidence: timing (the changes occurred at a time when the proliferation of personal computers and networking was highest), and the simple observation that high-skilled work is best suited to take advantage of the new technology. One way of testing the association between technology and wage changes is direct ethnographic research, conducted at the firm level, discussed above (\S\ref{sec:ethnographic}).

Unless the data can be augmented to include some measure of the properties of occupations coded in survey data, econometric analysis cannot draw conclusions about the types of jobs that are most affected by technological change. We have seen above (\S\ref{sec:upskillingau}) that `skill' data in occupational classifications can inform analysis to some degree, but this information does not discriminate between the types of skills that are impacted by changing technology. Fortunately, the occupational classification schemes provided by the US Department of Labor, published as {\em The Dictionary of Occupational Titles} (DOT) between 1939 and the 1990s, and {\em O*NET}, an electronic database, first released in 1998, {\em do} include detailed `task' information, along with occupational titles. The 2010 edition of the O*NET includes a taxonomy of 921 occupations, as well as detailed information about each of these occupations on a large number of quantitative scales. A more detailed discussion of the O*NET data can be found in the appendix (\S\ref{sec:onet}).

The use of the DOT and O*NET database as the basis for analytical studies of the work force is not new, nor is it exclusive to the Economics literature. Sociologists \citet{Cain1981} review a considerable sociological literature that employs the DOT's quantitative scales to analyze changes in the wage distribution. They show that, although there is considerable redundancy in the DOT's 44 measures, and although certain job characteristics (such as authority relationships and seniority) are missing, they contain at least as much information as many scales built specifically for the purposes of sociological analysis.

The detailed job criteria available in the DOT and O*NET have been exploited to explore the relationship between jobs' characteristics, and changes in the both the share of employment and the wage profile.
\citet{Levy2003} use the DOT to construct indexes for `routine' and `cognitive' components of jobs, that they regress on employment levels and wages across industries in the United States. They show that this model explains a considerable proportion of the dispersion of wages in the United States between 1960 and 1998, and that computerization led to a substitution in the observed levels of employment, away from routine tasks and toward cognitive tasks. 

The O*NET data have been exploited in foreign jurisdictions as well. By mapping UK job codes to O*NET codes, \citet{Goos2007} find a similar trend in the United Kingdom: between 1975 and 2003, there was an increase in the number of high-skilled, high-wage (which they dub `lovely') jobs, as well as low-wage, low-skilled (`lousy') jobs, but a relative decrease in the number of `middling' jobs. In a subsequent paper, a similar pattern was found for Continental Europe \citep{Goos2009}.

For Australian data, \citet{Esposto2011} performed a mapping between the O*NET classfication and the ASCO II. Using the O*NET data, they  construct a `knowledge intensity' index, which they take to be a proxy for `skill.' They find that the Australian work force, overall, has increased in its knowledge intensity between 1971 and 2006. However, they also find that the distribution of knowledge is shifting: away for men and toward women, and away from part-time workers and toward full-timers.

\section{Wage Profile Decompositions}

The evidence considered above suggests that, over time, the skill distribution of both the US and Australian populations has been shifting. A greater proportion of both populations has attained tertiary degrees, for example, and women's work force participation patterns have changed. This presents a problem for comparing wage profiles over time. A direct analysis of the wage profile, without knowledge of changing composition of the work force, cannot determine whether any observed changes occurred as a result of changing human capital variables, such as experience and educational attainment, or as a result of structural factors, such as technological change \citep[][see, e.g.]{Mincer1974}. 

\subsection{A Reweighting Approach}\label{sec:reweight}

Reweighting techniques overcome the problem of composition effects by computing a `counterfactual' distribution, that has the same distribution of covariates as the comparison distribution. First, suppose we have a set of observations, in which individuals can either be observed in period $0$ or $1$.

The goal of this approach is to re-weight the observations in period $0$ so that the covariates in period $0$ match those in period $1$. Adopting the re-weighting procedure suggested by \citet{DiNardo1996}, they aim to create a counterfactual wage distribution $F_{Y_0}^C$ that exhibits the characteristics of period $0$, but with the wage structure of period $1$:
\begin{align*}
  F_{Y_0}^C &= \int F_{Y_0|X_0}(y|X) dF_{X_1}(X)
\intertext{We now re-write this equation as an integral over $F_{X_0}(X)$, by adding a reweighting factor $\Psi(X) = dF_{X_1}(X)/dF_{X_0}(X)$:}
  F_{Y_0}^C &= \int F_{Y_0|X_0}(y|X) \Psi(X)dF_{X_0}(X)
\end{align*}
\citet{DiNardo1996} show that this re-weighting factor, which is the ratio of two marginal distribution functions, can be manipulated with an application of Bayes' rule to yield a ratio of two binary outcome models:
\begin{align*}
  \label{eq:wt}
  \Psi(X) &= \frac{\Pr(T=1|X)/\Pr(T=1)}{\Pr(T=0|X)/\Pr(T=0)},
\end{align*}
that re-weights the data in period $0$ to match the distribution of covariates observed in period~$1$. To implement this re-weighting function, the probability of $T$ being 1 or 0 can be modeled using a probit model, fit to the combined data sets, with $T$ as the response variable.

The \citet{DiNardo1996} approach to decomposition by re-weighting as been used in the Australian context by \citet{Baron2010}, who decompose the gender wage gap measured in the HILDA database into a difference explained by wage-related characteristics, and a component that is unexplained.

\subsection{The Oaxaca-Blinder Decomposition}

The Oaxaca-Blinder decomposition allows changes in the wage distribution to be attributed to a set of covariates that impact upon wages. The decomposition methods described here were first described in separate papers by \citet{Oaxaca1973} and \citet{Blinder1973}. 

Consider some outcome variable, such as an average log wage, that differs for two disjoint groups. Oaxaca, for instance, considered the difference in mean wages paid to men and women. Let the difference in the mean wage for men and women be $\Delta$:
\begin{equation} \Delta_O = E[\ln y_m] - E[\ln y_f]. \label{eq:odec} \end{equation}
If $\Delta$ is nonzero, this might be explainable by (a) factors arising from different human capital endowments in each group, (b) factors arising purely from group membership, or (c) both. The goal of the Oaxaca-Blinder (OB) decomposition is to divide this difference into two components: the component explainable by human capital factors (the endowment effect), and a structural component attributable only to group membership.

To determine the influence of sex on the mean of the wage distribution, Oaxaca considered two separate regression models, one for each sex. Each vector $X_i$ of covariates included demographic and human capital variables such as years of education, work experience and age:
$$  \ln y_{g,i} = \ve{X}_{g,i}'\vbeta_g + \epsilon_{g,i} \quad \text{where}\ g=M,F. $$
Then, taking expectations of both sides and substituting into \eqref{eq:odec}, the difference of expected log wages can be decomposed as,
\begin{align}
  \Delta_O &= E[X_m]'\vbeta_m -  E[X_f]'\vbeta_f \notag \\
  &= \underbrace{E[X_m]'(\vbeta_m - \vbeta_f)}_{\Delta_S} + \underbrace{(E[X_m]'-E[X_f]')\vbeta_f}_{\Delta_X}. \label{eq:odecomp}
\end{align}
The second term of this decomposition, $\Delta_X$, is the difference in mean log wages that can be explained by human capital factors (the `endowments effect'). The other term, $\Delta_S$, represents the `structural' difference in wages between the two groups. In the case where the wages of males and females are being considered, this term can be interpreted as the sex discrimination differential. The parameters in \eqref{eq:odecomp} are computed at their means to determine the difference $E[X_m]'(\vbeta_m - \vbeta_f)$ attributable to discrimination, in the mean log wage.

In the SBTC literature, the object of interest is the distribution of wages, rather than differences in the conditional mean, and the two groups of interest are not gender groups, but rather two different time periods, at the start and end of the period during which technical change is suspected to have occurred. For simplicity, we refer to these time periods as $T=0$ and $T=1$, respectively.

\subsection{Unconditional Quantile Regression}

One major shortcoming of the Oaxaca-Blinder decomposition is that only the conditional means of a wage distribution, $E(Y|X)$, and its counterfactual can be compared. Recall that, in the Roy model described above, changes in the profitability of any occupation should result in the more efficient individuals self-selecting out of an occupation. The mean of a wage distribution is a poor instrument for observing this phenomenon: rather, any polarisation effect will be observed in the overall {\em distribution} of wages, $F_Y$. Furthermore, it may be that certain effects only occur in some parts of the wage distribution, so that measuring the distribution mean is not appropriate. Ideally, we would like to compute a decomposition similar to \eqref{eq:odecomp}, but which decomposes changes in the $\tau$th quantile of the wage distribution, $q_\tau(F_Y)$. Such a decomposition was considered by \citet{Firpo2011}; it is their technique, as described in \citet{Firpo2009}, that we apply here.

Under our decomposition, the wage of an individual $i$ is observed in one of two periods, $T=0$ or $T=1$. Under the hypothesis of wage polarisation, we will assume that individuals are paid under two distinct wage structures: the pre-polarisation wage structure that has distribution $F_{Y_0}$ (when $T=0$) and the post-polarisation wage structure, $F_{Y_1}$ (when $T=1$). We wish to decompose the observed overall change $\Delta^\tau$ in the quantile statistic, attributable to changes in work force composition $\Delta^\tau_X$ and changes in the wage structure, $\Delta^\tau_S$:
\begin{align}
  \Delta^\tau_O &= q_\tau(F_{Y_1|T=1}) - q_\tau(F_{Y_0|T=0}) \notag \\
  &= \underbrace{q_\tau(F_{Y_1|T=1}) -  q_\tau(F_{Y_0|T=1})}_{\Delta^\tau_S} + \underbrace{q_\tau(F_{Y_0|T=1}) - q_\tau(F_{Y_0|T=0})}_{\Delta^\tau_X} \label{eq:decomp}
\end{align}Notice that this decomposition depends on the availability of a hypothetical counterfactual distribution, $F_{Y_0|T=1}$, wherein the workers of period $1$ are paid according to the wage structure of period~$0$. Although such a distribution cannot be directly observed, \citet{Firpo2011} show that a consistent estimator of $F_{Y_0|T=1}$ can be found by re-weighting $F_{Y_0}$ to have the same distribution as $F_{Y_1}$.

\citet{Firpo2009} demonstrate that the aggregate decomposition \eqref{eq:decomp}, can be performed using an OLS regression on the recentered influence function of the distributional statistic in question.\footnote{Note that RIF regressions must be used in the Oaxaca-Blinder decomposition, and not quantile regressions, because the law of iterated expectations only holds for the conditional mean of a distribution, and not other functionals of the distribution. See \citet{Firpo2009} for a detailed discussion.} The recentered influence function is the usual influence function used in the analysis of robust estimators, `recentered' by adding back the value of the distributional statistic. In the case of the quantile function $q_\tau$, the RIF is given by,
$$ RIF(y; q_\tau) = q_\tau + IF(y; q_\tau) = q_\tau + \frac{q_\tau - \mathbf{1}\{y \leq q_\tau\}}{f_Y(q_\tau)}. $$

Then the estimated coefficient $\gamma^{q_\tau}_t$ of an OLS regression of $RIF(y; q_\tau)$ on the set of wage-related characteristics, $X$, is
\begin{align*} 
\gamma^{q_\tau}_t &= (E[X \cdot X' | T = t])^{-1} \cdot E[RIF(y_t; q_\tau) \cdot X | T = t]
\intertext{\citet{Firpo2009} show that the distributional statistics themselves can be written as expectations of the conditional RIF, since the expected value of the influence function is zero, and thus $E[RIF(y_t;q_\tau)]=q_\tau$:}
q_\tau(F_t) &= E_X[E[RIF(y_t; q_\tau) | X=x]] = E[X|T=t] \cdot \gamma^{q_\tau}_t,
\intertext{And thus we can write \eqref{eq:decomp} in a similar form as \eqref{eq:odecomp}:}
\Delta^\tau_O &= \underbrace{E[X|T=1] \cdot (\gamma^{q_\tau}_1 - \gamma^{q_\tau}_0)}_{\Delta^\tau_S} + \underbrace{(E[X|T=1] - E[X|T=0]) \cdot \gamma^{q_\tau}_0}_{\Delta^\tau_X}.
\end{align*}
Under the `ignorability' assumption, discussed in the following chapter (\S\ref{sec:id}), both of these components of the decomposition are identified.



\subsection{Hybrid Approaches}\label{sec:reweight}

\citet[p.19]{Firpo2011} point out that the RIF-regression described above is a local approximation that may not hold for large variations in covariates $X$. In particular, if the relationship between $Y$ and $X$ is nonlinear, then shifts in the distribution of $X$ may result in different estimates for $\gamma^{q_\tau}_t$ even if $Y$ is invariant. 

Unfortunately, for labor force data stretching over a decade or even an entire generation, changes in covariates between period $T=0$ and $T=1$ cannot be assumed to be small. ABS data show that there are considerable differences in the composition of the labor force between 1981-2 and 2009-10 \citep{LFSApr2013}. The average unemployment rate in 1981-2 similar to that of 2009-10 (6.1 per cent versus 5.7 per cent, respectively), but the period was marked by considerable demographic changes. Since the 1980s, women have entered the work force in far greater numbers, and overall labor force participation patterns have varied. Between 1981-2 and 2009-10, the average participation rate for men fell from 77.7 per cent to 72.3 per cent. For women, on the other hand, the participation rate rose from 44.8 per cent to 58.6 per cent. And, for both sexes, the rate of part-time employment has increased dramatically. Clearly, the covariate distributions at both time periods are not directly comparable.

Using re-weighted data, we can estimate the means of the counterfactual distribution, $\hat{\bar{X}}=\sum_{i|T=0}\hat{\Psi}(X_i) \cdot X_i$, and the coefficients $\hat{\gamma}_{01}^{q_\tau}$ by regressing $RIF(Y_0;q_\tau)$ with the new sample weights. We then rewrite the decomposition \eqref{eq:decomp} as the sum of two separate Oaxaca-Blinder decompositions. The first term, the wage structure effect, is decomposed into a composition effect $\hat{\Delta}^{q_\tau}_{S,p}$ and specification error, $\hat{\Delta}^{q_\tau}_{S,e}$. The second gives a similar decomposition for the composition effect:
\begin{align*}
  \hat{\Delta}^{q_\tau} &= (\hat{\Delta}^{q_\tau}_{S,p} + \hat{\Delta}^{q_\tau}_{S,e}) + (\hat{\Delta}^{q_\tau}_{X,p} + \hat{\Delta}^{q_\tau}_{X,e}) \notag \\
  &= \underbrace{\left( [\bar{X}_{01} - \bar{X}_0 ] \hat{\gamma}^{q_\tau}_{01} +
    \bar{X}_{01}[\hat{\gamma}_{01}^{q_\tau} - \hat{\gamma}_0^{q_\tau}] \right)}_{\hat{\Delta}^{q_\tau}_{S}} +
  \underbrace{\left( \bar{X}_{1}[\hat{\gamma}_{1}^{q_\tau} - \hat{\gamma}_{01}^{q_\tau}] + 
    [\bar{X}_{1} - \bar{X}_{01} ] \hat{\gamma}^{q_\tau}_{01}\right)}_{\hat{\Delta}^{q_\tau}_{X}}.
\end{align*}
This decomposition can be performed on income surveys of repeated cross-sections of the same markets over time. \citet{Firpo2011} apply this technique to several pairs of cross sections between 1976 and 2010. By including occupational task measures in their set of explanatory variables, they are able to decompose changes in the unexplained portion of the wage distribution changes according to whether a job is susceptible to technological change, and the degree to which that job can be offshored. They find that technology was skill-biased during the 1980s, affected by off-shoring in the 1990s, but that from the 2000s technology effects were no longer observed.

%\citet{Baron2010} used this approach on the HILDA data.

\section{Other Approaches}

Using cointegration techniques, \citet{Gaston2009} incorporate \citet{Leigh2005}'s income tax data in a time series model of the relationship between the Gini coefficient and macroeconomic variables, including the terms of trade, investment in ICT infrastructure, the unionisation rate, and indexes of social and economic globalisation. By applying restrictions to the resulting time series model, they are able to test Granger (non-)causality between indexes of technological change and measurements of inequality. Along with other globalization indexes, they find that technology investment, interpreted as a proxy for SBTC, Granger causes increases in the Gini coefficient. They conclude, therefore, that firm investment in new technology is contributing to a general increase in income inequality.

\section{Summary}

In this chapter, we reviewed the main empirical treatments of technological change, with an emphasis on Australian studies. The consensus in the literature is that, like other developed nations, Australia is experiencing technological change, and that this change has manifested in work force upskilling, particularly in the 1980s and 1990s.

There is little evidence to date that technological change is causing widening of the skill spectrum, a theory that has found wide acceptance for other industrialized countries. Rather, studies generally agree that as firms have shifted towards skilled labor, the supply of skilled workers has evenly kept pace with demand.

%%% Local Variables: 
%%% mode: latex
%%% TeX-master: "paper"
%%% End: 
