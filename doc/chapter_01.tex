\chapter{Introduction}

This thesis is motivated by the question of whether technological change in the workplace increases income inequality. We depart from the `canonical' neoclassical model of skill-based technological change, which has not been able to explain empirical regularities of income distribution changes in industrialized nations in recent decades. Instead, we test whether a model of the \emph{task content} of workers' skills, of the type proposed by \citet{Levy2003}, can explain the changing nature of the Australian workforce.

The second half of the 20th century has witnessed tremendous change for Australian workers. Since 1973, average real per capita incomes have approximately doubled \citep{NA20124}. Economic growth has created over three million jobs \citep{LFSApr2013}. But the same period bore witness to a dramatic change in the distribution of incomes: in Australia, as well as in most developed countries, top percentile wage growth far outstripped that of lower-wage earners \citep{Atkinson1997,Borland1999}. Although income inequality in Australia fell somewhat between the 1950's and 1970's, it has since risen consistently for the last 30 years \citep{Leigh2005,Gaston2009}.

A leading explanation for this divergence of incomes is that new workplace technologies exhibit \emph{skill bias}, and disproportionately complement highly-skilled technical and managerial jobs \citep{Griliches1969,Autor2006}. Under this explanation, wages for high-skilled jobs increase as a result of an increase in the return to skilled labor, with demand for workers outstripping the supply. Likewise, as the relative demand for lower-skilled workers has softened, so relative wage growth has stagnated. 

This model, which has sparked a voluminous literature, has enjoyed considerable empirical success explaining rising wages for high-skill managerial and professional jobs in the United States and Europe \citep{Katz1992}. Since the canonical model includes \emph{factor-augmenting} capital, it predicts a uniform skill upgrading of the work force at all education levels \citep{Levy2003}. Skill upgrading has been confirmed by a number of authors, both in Australia \citep{Esposto2012, Wooden2000, Cully1999} and overseas \citep{Autor2008}. 

The canonical model also predicts a rising premium for high-skill workers. In the United States in particular, SBTC has been able to explain the dynamics of the wage premium demanded by tertiary-educated labor, which fell in the 1970s and has risen in the decades to 2008 \citep{Acemoglu2011}. However, the model substantially \emph{over-predicts} the magnitude of this differential for the United States \citep{Autor2008}. In Australia, a corresponding growth in the premium for tertiary qualifications has not been observed \citep{Coelli2009}.

There are, however, a number of empirical regularities that the canonical model fails to explain. Since the late 1990s, both in Europe and the United States, the data show a marked polarization in the work force \citep{Goos2007, Autor2006}. This polarization has simultaneously manifested in \emph{wages} and in \emph{jobs}: both wage growth and growth in the level of employment are concentrated in high-skill jobs, to a lesser extent, the bottom end of the skill spectrum. Middle-skill jobs have stagnated since the 1990s, both in terms of remuneration and level.

This uneven pattern of job growth suggests that some property of middle-skill jobs, not present in low- and high-skill jobs, is responsible for this observed stagnation. In order to understand these differences, a new analytical framework is required.

\section{The `Task Approach'}

The neoclassical production function, which views aggregate economic output as a simple function of inputs such as capital and labor, does not consider the specifics of the processes which produced that output \citep{Acemoglu2011}. Although the canonical approach has been very successful in explaining aggregate output levels, it is not sensitive to qualitative changes in the nature of production such as changes in the technology which produce output. 

% note: production function appropriate for manufacturing; not service economy

The {\em task approach}, a research program initiated by \citet{Levy2003}, presents an alternative perspective to the standard neoclassical production function. Rather than viewing output as a direct function of resource inputs, it separates the tasks performed by labor and technology, allowing  substitutions between factors \citep{Autor2013,Acemoglu2011}. 

The task approach facilitates the inclusion of worker \emph{skills} in model. For the purposes of this analysis, we follow \citet{Autor2013} in viewing a \emph{task} as a discrete unit of work, which may be used to create final goods and services, and a worker's \emph{skill}, as the stock of capabilities for the execution of those tasks. Importantly, under this framework, the allocation of workers' skills to tasks is considered endogenous to the model: heterogeneous workers apply their skills to tasks where they enjoy a competitive advantage.

Under this framework, the performance of tasks is not confined to human workers. Since the industrial revolution, investments in labor-saving capital by firms has seen a dramatic change in the performance of repetitive tasks. The pace of technical change has been continual: as automated looms replaced hand-weavers in the 18th century, so too are cheap computers replacing administrative clerks and service workers in the 21st century.

The level and price of task-performing labor can be 
viewed as an outcome of the demand for particular tasks from workers and machine capital, and the supply of task-performing labor and capital. Unlike the canonical model, where technology is viewed as factor-augmenting,  technology can therefore be viewed as substitutes for some tasks, and complements for others. Thus firms are able to substitute between capital and human workers for the execution of tasks.

\section{ICT and Routinization}

In recent decades, the most important source of labor-saving capital has been information and computer technology (ICT). As the real cost of computation has fallen precipitously over the 20th century, computers have been able to execute a wider range of tasks at a lower cost. In the presence of falling costs of ICT, the question of work force polarization can thus be framed as an outcome of a decline in the real cost of computing capital, relative to the wage cost of human workers performing similar tasks.

Computers, despite their sophistication, are only capable of performing a very limited set of simple, routine tasks. They excel at processes which require calculation and simple symbolic manipulation, and are not prone to the same types of errors as human workers. It is this fact which has led to their widespread adoption in automated tellers and a wide range of electronic service delivery which were formerly the domain of human personnel. Yet, any task that requires abstract thought or physical coordination, however elementary they may appear to a human worker, is out of reach for a computer. Activities such as stacking shelves or driving a taxi are areas in which, for the moment at least, human workers enjoy a competitive advantage \citet{Levy2003}. 

Non-routine tasks, on the other hand, may improve, rather than replace, the efficiency of human workers. Indeed, as \citet{Borland2004} found by studying the computer knowledge of a cross-section of Australian workers surveyed in 1992, computer knowledge accrues a skill premium of around 10\%.

Thus computing capital is a complement to some kinds of task-performing labor, and a substitute for others. As \citet{Levy2003} show, in the United States between 1960 and 1998, computerization led to a substitution in the observed levels of employment, away from routine tasks and toward cognitive tasks. Likewise, \citet{Goos2007} show a similar trend in the United Kingdom: between 1975 and 2003, they find a increase in the number of ``lovely'' (high-skill, high-wage) jobs and ``lousy'' (low-wage, low-skill) jobs, but a relative decrease in the number of ``middling'' jobs. In a subsequent paper, a similar pattern was found for Continental Europe \citep{Goos2009}.

It is therefore plausible, that the widespread adoption of ICT is a major driving force behind compositional changes in the workforce. 

\section{Road map and Contribution}

There is already a vast literature studying the rise of wage inequality in Australia. Empirical studies have confirmed that both individual-level and household-level inequality have been rising since the 1980s \citep{Borland1999,Leigh2005,Gaston2009}. A number of studies exist on the task content of Australian jobs \citep{Esposto2012a}, and the change over time of the skill intensity of various professions \citep{Esposto2012, Esposto2012a}. Although ICT use and globalization have been found to (non-)Granger cause rising inequality \citep{Gaston2009}, no studies have tested whether workers' skill allocation is the channel through which this change has occurred.

This paper aims to test the hypothesis that the deployment of ICT capital has displaced workers in routine jobs. We operationalize the model of \citep{Acemoglu2011} to test whether the skills channel is a mediator for rising inequality through a rising use of ICT.

%%% Local Variables: 
%%% mode: latex
%%% TeX-master: "thesis"
%%% End: 
