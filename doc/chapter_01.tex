\chapter{Introduction}

Over the second half of the 20th century, global integration and technological advance have driven tremendous change in Australia. Since 1973, average real income per capita has approximately doubled \citep{NA20124}. And as it has grown, the economy has added over three million jobs between 1978 and 2013 \citep{LFSApr2013}. This period also saw a qualitative change in the nature of jobs available, and the skills those jobs require \citep{Esposto2012a}.

% general trends

Wage inequality rising since 1960s, especially in the top half of the earnings distribution. In the United States, where real wages for unskilled workers have declined, on average, since 1970, the skill premium for college graduates has grown significantly. But changing distribution of education doesn't explain the rise in inequality. Robust to industry (shown with census data.)

Causes of SBTC include (a) cheaper capital/computers, (b) institutional changes. Skills can be considered (a) complement physical+computer capital, or (b) particular skills needed for rapid change. Other explanations incude international openness and trade, increasing competition for jobs at low end of spectrum. Further, labor institutions are changing, especially weaker unions.
                
US, Canadian and British experience particularly pronounced. Europe experienced similar increase in inequality; borne out more in unemployment than wage (institutional explanation.) The key explanation has been skill-based technical change, which is increasingly rapid due to the pace of technical advances. Supply of skilled workers sped up in 1970s, slowed in 1980s; under this explanation, excess of demand gives wage increase.

Industry-level evidence: all industries increase in skill demand, skill premium. Change more rapid in industries with increasing computerization.

The skill-based technical change hypothesis predicts a uniform skill upgrading of the workforce at all education levels. Although skill upgrading has been found, both in Australia \citet{Esposto2012}\citet{Wooden2000}\citet{Cully1999} and overseas \citet{Autor2008} by a number of authors, SBTC is unable to explain the polarization of job opportunities in the United States and Europe.

\section{The Task Approach}

The neoclassical production function, which views aggregate economic output as a simple function of inputs such as capital and labour, does not consider the specifics of the processes which produced that output. Although the production function approach has been very successful in explaining aggregate output levels, it is not sensitive to qualitative changes in the nature of production such as changes in the technology which produce output.

% note: production function appropriate for manufacturing; not service economy

The {\em task approach}, a research program initiated by \citet{Levy2003}, presents an alternative perspective to the standard neoclassical prodution function. Rather than viewing output as a direct function of resource inputs, it separates the tasks performed by labour and technology, allowing substitutions between factors \citep{Autor2013}.

The task appraoch facilitates the inclusion of worker \emph{skills} in model. For the purposes of this analysis, we follow \citet{Autor2013} in viewing a \emph{task} as a discrete unit of work, which may be used to create final goods and services, and a worker's \emph{skill}, as the stock of capabilities for the execution of those tasks. 

The performance of tasks is not confined to human workers. Since the industrial revolution, investments in labour-saving capital by firms has seen a dramatic change in the performance of tasks by workers. The pace of technical change has been continual: as automated looms replaced hand-weavers in the 18th century, so too are cheap computers replacing administrative clerks and service workers in the 21st century.

Under the task approach, the level and price of task-performing labour, can be 
viewed as an outcome of the demand for particular tasks from workers and machine capital, and the supply of task-performing labour and capital. Some types of human labour and computer capital can therefore be viewed as substitutes: firms are able to substitute between computers and human workers for the execution of tasks. 

As the real cost of computer capital has fallen precipitously over the 20th century, substitution between human and physical capital has increased. The question of polarization of the workforce can thus be modeled as an outcome of a decline in the real cost of computing capital, relative to the wage cost of human workers performing similar tasks.

Computers, despite their sophistication, are only capable of performing a very limited set of simple, routine tasks. They excel at processes which require calculation and simple symbolic manipulation, and are not prone to the same types of errors as human workers. It is this fact which has led to their widespread adoption in automated tellers and a wide range of electronic service delivery which were formerly the domain of human personnel. Yet, any task that requires abstract thought or physical coordination, however elementary they may appear to a human worker, is out of reach for a computer. Activities such as stacking shelves or driving a taxi are areas in which, for the moment at least, human workers enjoy a competitive advantage \citet{Levy2003}. 

Non-routine tasks, on the other hand, may improve, rather than replace, the efficiency of human workers. Indeed, as \citet{Borland2004} found by studying the computer knowledge of a cross-section of Australian workers surveyed in 1992, computer knowledge accrues a skill premium of around 10\%.

Thus computing capital is a complement to some kinds of task-performing labour, and a substitute for others. \cite{Levy2003}, in their seminal paper on the topic, show that, in the United States between 1960 and 1998, computerization led to a substitution in the observed levels of employment, away from routine tasks and toward cognitive tasks. Likewise, \citet{Goos2007} show a similar trend in the United Kingdom: between 1975 and 2003, they find a increase in the number of ``lovely'' (high-skill, high-wage) jobs and ``lousy'' (low-wage, low-skill) jobs, but a relative decrease in the number of ``middling'' jobs. In a subsequent paper, a similar pattern was found for Continental Europe \citep{Goos2009}.

In a study of the Australian workforce, \citet{Esposto2012} decomposed the Australian workforce by type of labour, and found that, over time, the labour force is upskilling, but that the trend depends on the category of work being performed. Between XXXX and XXXX, the demand for managerial and professional tasks increased, but over the same period, 

The task-assignment model allocates high (H), medium (M) and low (L) skilled inputs on a unit interval. Computerisation, due to decr in cost of computing power, in routine tasks displaces the H/M and M/L boundary. Wage of M decreases, wage of H and L increase due to q-complementarity.

Major within-data limitations. Key: changing composition of tasks within jobs. Subject to continual optimisation. More recent literature considers actual tasks in jobs through surveys.
        
Also, endogenous task choice not considered by literature; should not assume assignment to skills are predetermined.
        
Further, orthogonal category: "offshorability."

Skills data available: http://web.mit.edu/dautor/www


%%% Local Variables: 
%%% mode: latex
%%% TeX-master: "thesis"
%%% End: 
